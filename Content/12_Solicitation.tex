% !TEX root = ../main.tex
% !TEX program = lualatex
\documentclass[../main.tex]{subfiles}
\IfSubfilesClassLoaded{\externaldocument{\subfix{../build/main}}}{}
% =============================================================================
% CHAPTER 12: SOLICITATION
%
% DESCRIPTION: This chapter covers the solicitation preparation process,
%              including practitioner steps, checklists, and Navy-specific
%              overlays. It also discusses industry engagement, contract types,
%              and other related topics.
% =============================================================================
\begin{document}
\ifSubfilesClassLoaded{\chapter{EDO Study Guide}}{}
%====================
% Contents
%====================
% === SOLICITATION PREPARATION (EDO 3.2.2) ===
\section{Solicitation Preparation}
\minibib
\subsection{Summary}
\begin{description}
	\item[\textbf{\ac{ucf} scaffolding.}] Tailor the \ac{ucf} (Sections~A through M) so Section~C defines the requirement, Section~L tells offerors how to respond, and Section~M mirrors evaluation factors and the stated basis of award~\autocite{FAR,edo-3-2-2-solicitation-prep-2025}.
	\item[\textbf{Release readiness.}] Do not release the \ac{rfp} until the acquisition plan/strategy is approved, funds are certified, legal review is complete, synopsis rules are satisfied, and the source-selection teams are chartered~\autocite{FAR,NMCARS,edo-3-2-2-solicitation-prep-2025}.
	\item[\textbf{Communication discipline.}] \acp{rfi}, draft \acp{rfp}, industry days, and Q\&A all flow through the \ac{ko}; once released, clarifications must be shared with all offerors through written amendments or controlled exchanges~\autocite{FAR,edo-3-2-2-solicitation-prep-2025}.
\end{description}

\subsection{Practitioner Steps}
\begin{enumerate}
	\item Finalize the requirement package (spec~/\ac{sow}, \acp{cdrl}, \ac{igce}, market research report) with \ac{ipt} concurrence and Small Business coordination documentation~\autocite{FAR,edo-3-2-2-solicitation-prep-2025}.
	\item Structure Section~L instructions: volumes, page limits, submission medium, proposal due date, classification controls, and~\autocite[\S~5.203]{FAR} synopsis timing.
	\item Align Section~M evaluation factors, subfactors, and relative importance statements with the basis of award (\ac{lpta} v.s.\ trade-off) and rating methodology approved by the \ac{ssa}, \ac{ko}, legal, and cost/price community~\autocite{FAR,NAVSEA-Source-Selection-Guide-2022}.
	\item Verify pre-release approvals and postings: acquisition plan or strategy, Determination of Acquisition Strategy (as applicable), certified funds, and \url{https://www.SAM.gov} synopsis~\autocite{NMCARS,edo-3-2-2-solicitation-prep-2025}.
	\item Manage amendments and records after release: capture bidder Q\&A, issue conformed \acp{rfp}, adjust milestones when requirements shift, and file every action in the official contract record~\autocite{FAR,NAVSEA-Contracts-Handbook-2023}.
\end{enumerate}

\subsection{Checklist}
\begin{itemize}
	\item Acquisition plan/strategy signed; conformed requirements package appended~\autocite{NMCARS}.
	\item Funding document (\ac{pr}, project order, or \ac{mipr}) signed and funds certified for the planned obligation~\autocite{FAR}.
	\item Market research report and Small Business Coordination Record approved~\autocite{FAR}.
	\item Source Selection Plan endorsed by \ac{ssa}, legal, and cost/price; nondisclosure/\ac{oci} statements executed for \ac{ssa}, \ac{ssac}, \ac{sseb}, and advisors~\autocite{NAVSEA-Source-Selection-Guide-2022}.
	\item Section~L instructions mirror Section~M factors, include proposal structure, late proposal policy, and submission portal guidance~\autocite{FAR}.
	\item \url{https://www.SAM.gov} synopsis posted (15~days before closing for most supplies/services; 30~days for R\&D unless justified)~\autocite{FAR}.
	\item Amendment template/change log prepared; routing matrix established for expedited approvals~\autocite{NAVSEA-Contracts-Handbook-2023}.
\end{itemize}

\subsection{Navy Overlays and Tailoring}
\begin{description}
	\item[\textbf{\ac{nmcars} overlays.}] Follow~\autocite[\S~5205.303/5205.301]{NMCARS} for synopsis exceptions and \ac{navsea} contracting notices for Contract Requirement Package checklists and release approvals~\autocite{NMCARS,NAVSEA-Contracts-Handbook-2023}.
	\item[\textbf{Technical data.}] Coordinate with \ac{ta} on~\autocite[\S~252.227]{DFARS} clauses, distribution statements, and \ac{cdrl} tailoring to protect Navy equities.
	\item[\textbf{Shipbuilding tailoring.}] For availabilities, modernization, or planning efforts, ensure Section~C and the \ac{wbs} matches the avail sequence and integrates warfare-center technical direction and installation responsibilities~\autocite{edo-3-2-2-solicitation-prep-2025}.
\end{description}

\subsection{Industry Engagement \& Communications}
\begin{description}
	\item[\textbf{Pre-release.}] \acp{rfi}, sources-sought notices, draft \acp{rfp}, and industry days are vetted through the \ac{ko}; responses shared with all potential offerors to avoid unequal access~\autocite{FAR}.
	\item[\textbf{Post-release.}] Clarifications and Q\&A must be issued as amendments for all offerors; exchanges stay within~\autocite[\S~15.201]{FAR} boundaries until competitive range is established~\autocite{NAVSEA-Contracts-Handbook-2023}.
	\item[\textbf{Small business focus.}] Coordinate with the Small Business Professional and Competition Advocate on set-aside decisions, subcontracting plan reviews, and mitigation of potential limitations on subcontracting~\autocite{FAR}.
\end{description}

\subsection{Documentation \& Approvals to Capture}
\begin{itemize}
	\item Signed Source Selection Plan and \ac{ssa}/\ac{ssac}/\ac{sseb} appointment letters with confidentiality and \ac{oci} certifications~\autocite{NAVSEA-Source-Selection-Guide-2022}.
	\item Competition Advocate memoranda where narrowed sources or synopsis deviations are requested~\autocite{FAR,NMCARS}.
	\item Legal sufficiency memorandum for the conformed solicitation (Sections~C, H, L, and M) retained in the contract file~\autocite{NAVSEA-Contracts-Handbook-2023}.
	\item \ac{ko} release memorandum noting solicitation number, issue date, amendment log, and proposal due date/time~\autocite{edo-3-2-2-solicitation-prep-2025}.
\end{itemize}

\subsection{Competitive Method Selection: Sealed Bidding v.s.\ Negotiated Procurement}
\begin{description}
\item[\textbf{Sealed bidding (FAR Part~14).}] Preferred when requirements are well-defined, award will be based on price alone, discussions are not needed, and the Government expects to receive more than one responsive bid. It uses public opening, responsiveness determinations, and fixed-price awards~\autocite[Part~14]{FAR}.
\item[\textbf{Negotiated procurement (FAR Part~15).}] Used when the Government needs to evaluate technical approach or past performance, intends to conduct discussions, or when requirements or pricing demand flexibility. Negotiated acquisitions support best-value trade-offs and cost-reimbursement vehicles~\autocite[Part~15]{FAR}.
	\item[\textbf{When to choose.}] If the team can describe performance in terms of precise specifications with minimal risk and expects no need for exchanges, sealed bidding streamlines award. When innovation, risk, or integration complexity requires subjective evaluation, or when schedule/budget changes are likely, negotiated procurement is mandatory~\autocite{FAR,NAVSEA-Source-Selection-Guide-2022}.
\end{description}

\subsection{Contract Type Landscape and Risk Allocation}
\begin{description}
	\item[\textbf{Fixed-price family.}] Contractor accepts the cost risk; Government locks price once requirements are stable. Includes:
      \begin{itemize}
        \item \textbf{\ac{ffp}}: Used for mature, low-risk requirements, often in Production and Deployment~\autocite{FAR,DoDI5000-85}.
        \item \textbf{\ac{fpif}}: Shares cost variance through a negotiated share ratio; encourages cost control while keeping final price bounded by a ceiling. Used during Engineering \& Manufacturing Development (\ac{emd}) or \ac{lrip} when cost risk remains but production discipline is needed~\autocite{FAR,DoDI5000-85}.
        \item \textbf{\ac{fpepa}}: Adds indexed adjustments for volatile commodities or long-lead items~\autocite{FAR}.
        \item \textbf{\ac{fpr}}: Re-prices after defined milestones when initial uncertainty is expected to retire~\autocite{FAR}.
      \end{itemize}
        \begin{figure}[H]
          \centering\color{DarkGray}
          \includegraphics[width=.78\linewidth]{Images/FP_charts.png}
          \caption[Fixed Price Contracts]{Fixed-price incentive geometry (cost, profit, \glsentryshort{pta}).}
          \label{fig:fp_charts}
        \end{figure}
	\item[\textbf{Cost-reimbursement family.}] Government carries more cost risk; contractor must deliver best effort. Includes:
      \begin{itemize}
        \item \textbf{\ac{cpff}}: Used in early R\&D or prototyping (\ac{msa}/\ac{tmrr}) when effort is exploratory~\autocite{FAR,DoDI5000-85}.
        \item \textbf{\ac{cpif}}: Shares cost variance similar to \ac{fpif} but without a price ceiling; encourages efficiency while acknowledging uncertain cost baseline~\autocite{FAR}.
        \item \textbf{\ac{cpaf}}: Adds subjective award-fee evaluation for mission effectiveness or management performance that cannot be captured in objective metrics~\autocite{FAR}.
      \end{itemize}
      \begin{figure}[H]
        \centering\color{DarkGray}
        \includegraphics[width=.78\linewidth]{Images/CR_charts.png}
        \caption[Cost Reimbursement Contracts]{Cost-reimbursement contract risk/fee characteristics. \srcCite{edo-3-2-2-solicitation-prep-2025}.}
        \label{fig:cr_charts}
      \end{figure}
	\item[\textbf{Award v.s.\ incentive fees.}] Incentive fees (cost or performance) are calculated from predetermined formulas tied to measurable outcomes (cost, schedule, technical). Award fees are earned through periodic board evaluations against tailored factors and may be unearned if performance is merely satisfactory~\autocite{FAR,NAVSEA-Contracts-Handbook-2023}.
	\item[\textbf{Award-fee governance.}] The Award Fee Determining Official (often the \ac{pm} or \ac{ssa} designee) chairs an Award Fee Board, uses the approved plan, and issues a determination memo. Fees are paid after each period and must be commensurate with value delivered; scores below the threshold earn zero dollars for that segment~\autocite{FAR,NAVSEA-Source-Selection-Guide-2022}.
	\item[\textbf{\ac{mca} phase alignment.}] \ac{msa} and \ac{tmrr} generally use \ac{cpff} or \ac{cpaf} because of design uncertainty; \ac{emd} can transition to \ac{cpif} or \ac{fpif} as risk retires; Production and Deployment favors \ac{ffp}/\ac{fpif}; Operations and Support relies on \ac{ffp} or \ac{ffp} \ac{loe} and can incorporate sustainment \acp{idiq} for depot work~\autocite{DoDI5000-85}.
      \begin{itemize}
        \item[\textbf{MSA.}] Analytical trades and early prototypes benefit from \ac{cpff} flexibility as requirement scope evolves.
        \item[\textbf{TMRR.}] \ac{cpif}/\ac{cpaf} maintain incentives while the Government still absorbs major technical risk during competitive prototyping.
        \item[\textbf{EMD.}] \ac{fpif} or \ac{cpif} balance contractor motivation and cost control once the design stabilizes, with ceilings to protect the Government.
        \item[\textbf{PD.}] \ac{ffp}/\ac{fpif} dominate because production baselines and learning curves are known; contractor should own execution risk.
        \item[\textbf{OS.}] Sustainment leverages \ac{ffp}, \ac{ffp} \ac{loe}, or \acp{idiq} task orders to manage repeatable workloads and retain competition for modernization.
      \end{itemize}
      \begin{figure}[H]
        \centering\color{DarkGray}
        \includegraphics[width=.78\linewidth]{Images/contract_by_phase.png}
        \caption{Contract-type emphasis across the Major Capability Acquisition phases.}
        \label{fig:contract_by_phase}
      \end{figure}
\end{description}

\subsection{Fixed-Price Incentive (Firm Target) Mechanics}
\begin{description}
	\item[\textbf{Share ratio.}] Defines how cost overruns or underruns are split between Government and contractor (e.g., 70/30 means Government absorbs 70% of variance, contractor 30%)~\autocite{FAR}.
	\item[\textbf{Point of total assumption (\ac{pta}).}]
      The cost point beyond which every additional dollar of overrun is borne entirely by the contractor:
      \[
        \ac{pta} = \frac{\text{Ceiling Price} - \text{Target Cost}}{\text{Government Share Ratio}} + \text{Target Cost}.
      \]
\item[\textbf{Example.}] Target cost \$100M, target profit \$10M (target price \$110M), ceiling price \$120M, share ratio 70/30. If actual cost is \$112M:
      \[
        \text{Contract price} = \$110\text{M} + 0.3 \times (\$100\text{M} - \$112\text{M}) = \$106.4\text{M}.
      \]
      The \ac{pta} occurs at \(\$113.\overline{3}\,\text{M}\); beyond that, profit erodes dollar-for-dollar until exhausted~\autocite{FAR,NAVSEA-Contracts-Handbook-2023}.
\end{description}

\subsection{Indefinite \& Flexible Contract Forms}
\begin{description}
	\item[\textbf{\ac{idiq}.}] Indefinite-Delivery, Indefinite-Quantity contracts provide flexibility in quantity and timing with task/delivery orders; suited for recurring installations, sustainment, or services~\autocite{FAR}.
	\item[\textbf{\ac{tnm}/\ac{lh}.}] Hybrid vehicles when it is not possible to estimate effort precisely; require ceiling price, surveillance, and justification because Government assumes cost risk on labor hours~\autocite{FAR}.
	\item[\textbf{Undefinitized contract actions (\acp{uca}).}] Letter contracts or other actions authorized before all terms are settled (e.g., urgent ship repair). Must be definitized within the regulatory timeline, with strict management for obligation limits and profit~\autocite{FAR,DFARS,NMCARS}.
\end{description}

\subsection{Debriefings and Protests (Forums and Timelines)}
\begin{description}
  \item[\textbf{Debriefings.}] Post-award (and certain pre-award) debriefings are conducted per~\autocite[\S~15.506]{FAR}. For DoD enhanced debriefings, the question period extends the debrief conclusion for protest timeliness purposes; verify current Class Deviation guidance.
  \item[\textbf{Protest forums.}] (i) Agency-level to the contracting activity (\ac{ko}/Head of Contracting Activity)~\autocite[\S~33.103]{FAR}; (ii) \ac{gao} bid protests~\autocite[\S~33.104]{FAR}; (iii) U.S. Court of Federal Claims (COFC) under 28~U.S.C.~\S~1491(b).
  \item[\textbf{Who handles them.}] Agency: procuring activity and counsel; GAO: \ac{gao} decides, agency produces the record and implements any stay; COFC: Department of Justice litigates, COFC judge adjudicates and may grant injunctive relief.
  \item[\textbf{Timelines (common rules).}] Pre-award challenges to solicitation terms must be filed before the proposal due date. For post-award protests at \ac{gao}: file within 10~calendar days of when the basis is known, or within 5~business days after debriefing conclusion to obtain an automatic stay of performance under CICA. Agency-level protest timeliness follows similar 10-day knowledge rules in~\autocite[\S~33.103]{FAR}. COFC has no fixed day-limit but relief is equitable; prompt filing is expected.
\end{description}

\subsection{Why Cost-Plus After LRIP? (PD and O\&S Exceptions)}
\begin{description}
  \item[\textbf{High residual technical risk.}] Post-LRIP, major engineering change proposals, tech refresh, or obsolescence-driven redesign can reintroduce uncertainty better handled under \ac{cpif}/\ac{cpaf} than \ac{ffp}.
  \item[\textbf{Unstable scope or emergent defects.}] Corrective actions discovered during \ac{iote}/early fielding may require iterative investigation and rework where a best-effort \emph{cost-reimbursable} vehicle protects schedule and access to talent.
  \item[\textbf{Sustainment stand-up.}] Initial depot activation, organic transition, or complex \ac{pbl} standing up new measurements and data flows can warrant short-duration \ac{cpff}/\ac{cpaf} with incentives while processes stabilize.
  \item[\textbf{Data rights/technical baseline gaps.}] Where the Government lacks detailed technical data to price \ac{ffp} confidently, a temporary cost-plus arrangement with aggressive \ac{cdrl} deliverables can bridge to \ac{ffp}.
  \item[\textbf{Governance.}] Treat as exceptions with a plan to transition to \ac{ffp}/\ac{fpif} once uncertainty retires; document rationale in the Acquisition Strategy and obtain approvals consistent with~\autocite{FAR,DFARS}.
\end{description}

\subsection{Simplified Acquisition and Small Business Programs}
\begin{description}
	\item[\textbf{Simplified Acquisition Threshold (\ac{sat}).}]
	      Generally \$250{,}000 (higher in contingency or overseas circumstances). Benefits include streamlined competition, quotation-based awards, and mandatory small-business reservations above the micro-purchase threshold~\autocite{FAR}.
	\item[\textbf{Small Business Administration (\ac{sba}) programs.}] FAR Part~19~\autocite[Part~19]{FAR} implements 8(a), \ac{hubzone}, \ac{sdvosb}, and \ac{wosb} programs. The contracting officer must consider set-asides, sole-source thresholds, and subcontracting plans to meet \ac{don} goals.
\end{description}

\subsection{Alternative Acquisition Authorities}
\begin{description}
	\item[\textbf{Other Transaction (\ac{ot}) agreements.}] Authorized by 10~U.S.C.~\S~4021 (prototype) and \S~4022 (follow-on production) to rapidly engage nontraditional performers, provided cost share or significant participation criteria are met~\autocite{USC-4021-OTA,USC-4022-OTF}.
	\item[\textbf{Small Business Innovation Research (\ac{sbir}).}]
	      15~U.S.C.~\S~638 establishes Phase~I (feasibility), Phase~II (prototype), and Phase~III (commercialization/production). \ac{sbir} awards leverage R\&D funds to deliver Navy-unique technology with data rights protections for five years~\autocite{USC-638-SBIR}.
\end{description}

\subsection{Technical Data Rights Overview}
\begin{description}
	\item[\textbf{Unlimited rights.}] Government may use, disclose, and authorize others without restriction (typically for data developed exclusively at Government expense)~\autocite{DFARS}.
	\item[\textbf{Government purpose rights.}] Government may use or authorize contractors for Government purposes for five years, after which rights convert to unlimited. Applies to mixed-funding developments~\autocite{DFARS}.
	\item[\textbf{Limited/Restricted rights.}] Contractor retains control; Government use is constrained to internal purposes (limited for noncommercial technical data, restricted for commercial computer software). Negotiated licenses or specifically negotiated licenses can expand access~\autocite{DFARS}.
\end{description}

\subsection{Uniform Contract Format Essentials}
\begin{description}
	\item[\textbf{Sections A through H (contract).}] Cover Standard Form~33, supplies/services, packaging, inspection, delivery, special contract requirements, clauses, and attachments~\autocite{FAR}.
	\item[\textbf{Sections I through K (legal/representations).}] Incorporate FAR/DFARS clauses, representations, certifications, and instructions for completion~\autocite{FAR,DFARS}.
	\item[\textbf{Sections L and M (solicitation/Evaluation).}] Instructions to offerors, proposal structure, page limits, and evaluation criteria that must remain synchronized to any change to factors or relative order requires an amendment~\autocite{FAR,NAVSEA-Source-Selection-Guide-2022}.
\end{description}

%====================
% End of file
%====================
\ifSubfilesClassLoaded{
  \RenewDocumentCommand{\entryname}{}{\textbf{\color{Modern} Acronym}}
  \RenewDocumentCommand{\descriptionname}{}{\textbf{\color{Modern} Definition}}
  \printnoidxglossary[
    type=\acronymtype,
    title=Acronyms,
    style=long-booktabs
]}{}
\end{document}
