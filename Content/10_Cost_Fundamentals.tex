% !TEX root = ../main.tex
% !TEX program = lualatex
\documentclass[../main.tex]{subfiles}
\IfSubfilesClassLoaded{\externaldocument{\subfix{../build/main}}}{}
% =============================================================================
% CHAPTER 10: COST FUNDAMENTALS
%
% DESCRIPTION: This chapter covers the fundamentals of cost estimating,
%              including cost elements, program cost terms, estimating
%              methods, and the learning curve.
% =============================================================================
\begin{document}
\ifSubfilesClassLoaded{\chapter{EDO Study Guide}}{}
%====================
% Contents here
%====================

% === COST FUNDAMENTALS ===
\section{Cost Elements, Estimates, and Learning Curves}
\minibib

\subsection{Direct v.s.\ Indirect Costs}
\begin{description}
	\item[\textbf{Direct.}] Traceable to a final cost objective (e.g., technician hours, end-item material).
	\item[\textbf{Indirect.}] Pooled and allocated (overhead, G\&A, facilities, depreciation). See the \ac{far} cost principles~\autocite{FAR}.
\end{description}

\subsection{Program Cost Terms (Know the Stack)}
\begin{description}
	\item[\textbf{Flyaway Cost.}] Recurring production cost of an aircraft/weapon delivered (excludes non-recurring tooling, spares, support).
	\item[\textbf{Weapon System Cost.}] Flyaway plus support equipment, training, data, and initial spares.
	\item[\textbf{Procurement Cost.}] Weapon system cost plus peculiar support and \ac{milcon} as applicable.
	\item[\textbf{\ac{lcc}.}] All costs from concept through disposal (\ac{rdte}, procurement, \ac{os.phase}, disposal).
	\item[\textbf{\ac{toc}.}] \ac{lcc} plus broader infrastructure/indirect enterprise costs.
\end{description}
\note{Use \ac{don}/\ac{dod} cost handbook definitions; be consistent across documents.~\autocite{SECNAV5000-2G,DoDI5000-85}.}

\Fig{\ref{fig:lcc_composition}} shows the \ac{lcc} from above and how they relate to one another.  Note how each cost builds to the next as different types of expenditures are added.

\begin{figure}[H]
	\centering
	\includegraphics[width=0.7\linewidth]{LCC_Composition.png}
	\caption[Life Cycle Costs]{Composition of Life Cycle Costs. \srcCite{edo-3-1-6-cost-estimating-2024}.}\label{fig:lcc_composition}
\end{figure}

\subsection{Why the LCCE Matters}
Establishes the affordability baseline, informs \ac{apb}, drives color-of-money phasing, and underpins ``will-cost / should-cost'' targets~\autocite{DoDI5000-85}. \Fig{\ref{fig:lcc_time}} shows the different types of funding that will be used in different phases of the \ac{mca}.  Most of the \ac{lcc} will be from the \ac{os.phase} phase.

\begin{figure}[H]
	\centering
	\includegraphics[width=0.7\linewidth]{lcc_over_time.png}
	\caption[Lif-Cycle Cost]{\ac{lcc} over time.  The colors of money used are shown for each phase including percent of total costs. \srcCite{edo-3-1-6-cost-estimating-2024}}
	\label{fig:lcc_time}
\end{figure}

\subsection{Estimating Methods (Types, When, and Why)}
\begin{description}
	\item[\textbf{Analogy (top-down).}] \emph{When:} Earliest stages (\ac{icd}/\ac{mdd}, pre- \ac{aoa}) when design is immature and only analogous programs or components exist. \emph{Why:} Fast, defensible scoping by scaling from a close comparator using adjustment factors (size, weight, speed, complexity). Good for bounding affordability and informing trades before detailed data exists.
	\item[\textbf{Parametric (CER-based).}] \emph{When:} \ac{aoa} through \ac{tmrr} when historical data supports \acp{cer} (e.g., weight, \acp{sloc}, thrust, kW). \emph{Why:} Enables rapid sensitivity analysis and trade-space exploration with statistical confidence intervals; supports independent cross-checks and early budget phasing.
	\item[\textbf{Engineering (bottom-up).}] \emph{When:} \ac{emd} and production planning when \acp{bom}, routing, labor standards, and quotes are available. \emph{Why:} Most traceable and auditable; ties to technical baseline and schedule networks; supports \acp{ibr} and negotiation objectives.
	\item[\textbf{Actuals / Extrapolation.}] \emph{When:} From late \ac{tmrr} into production when pilots/LRIP actuals exist and learning effects are visible. \emph{Why:} Highest fidelity; captures learning curves and process improvements; anchors estimates to realized performance and \ac{csdr} actuals~\autocite{DAU-AAF}.
\end{description}

\begin{center}
  \begin{talltblr}{
      row{1} = {font=\bfseries\scshape\color{Modern}},
      colspec = {@{} F L L L @{}}
    }
    \toprule
    {Type} & {Description} & {When Used} & {Why Used} \\
    \midrule
    Analogy & Uses cost of a similar past system, adjusted for differences & Early concept/refinement (Pre-Milestone A/B) & Quick estimate when data limited \\
    Parametric & Uses statistical cost estimating relationships (CERs) like cost vs. weight or power & Milestone B/C; early design & Captures cost drivers using quantifiable parameters \\
    Engineering (Bottom-Up) & Builds from detailed WBS---materials, labor, overhead & Post-CDR, LRIP & High accuracy for mature designs \\
    Actual Cost / Extrapolation & Uses historical/production actuals (learning curves) & FRP / O\&S & Most accurate; supports should-cost and sustainment budgeting \\
    \bottomrule
  \end{talltblr}
\end{center}

\subsection{Products and Players}
\begin{description}
	\item[\textbf{\ac{card}.}] Technical/schedule baseline for independent estimates.
	\item[\textbf{\ac{ice}.}] \ac{cae} or \ac{osd} \ac{cape} for \ac{acat} I; \ac{poe}.
	\item[\textbf{\ac{csdr}.}] Contractor cost \& software data reporting for actuals/\acp{cer}.
	\item[\textbf{\ac{apb}/\ac{sar}/\ac{daes}.}] Baselines and reporting~\autocite{DoDI5000-85}.
\end{description}

\subsection{Learning Curve (Production Efficiency)}
If unit time/cost follows an $s$\% slope, doubling quantity reduces unit cost by $s$\%. Typical aerospace slopes: 85--90\% early, flattening as processes mature. Plan lots and spares buys accordingly~\autocite{DAU-AAF}.

\subsection{Cost Escalation (Inflation Indices)}
Escalate base-year costs to then-year using approved \ac{dod} inflation indices; separate real growth from price growth; align phasing to obligation windows~\autocite{DoDFMR-Vol3}.

\subsection{Will-Cost / Should-Cost}
\begin{description}
	\item[\textbf{Will-Cost.}] Conservative reference estimate used for budgeting and margin planning.
	\item[\textbf{Should-Cost.}] Management targets that drive efficiencies (e.g., lean events, value engineering, rate negotiations)~\autocite{DoDI5000-85}.
\end{description}

%====================
% End of file
%====================
\ifSubfilesClassLoaded{
  \RenewDocumentCommand{\entryname}{}{\textbf{\color{Modern} Acronym}}
  \RenewDocumentCommand{\descriptionname}{}{\textbf{\color{Modern} Definition}}
  \printnoidxglossary[
    type=\acronymtype,
    title=Acronyms,
    style=long-booktabs
]}{}
\end{document}
