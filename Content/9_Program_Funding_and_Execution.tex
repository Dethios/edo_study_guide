% !TEX root = ../main.tex
% !TEX program = lualatex
\documentclass[../main.tex]{subfiles}
\IfSubfilesClassLoaded{\externaldocument{\subfix{../build/main}}}{}
% =============================================================================
% CHAPTER 9: PROGRAM FUNDING AND EXECUTION
%
% DESCRIPTION: This chapter covers program funding and execution, including
%              appropriation categories, funding policies, and financial
%              management at the field activity level.
% =============================================================================
\begin{document}
\ifSubfilesClassLoaded{\chapter{EDO Study Guide}}{}
%====================
% Contents here
%====================

% === PROGRAM FUNDING & EXECUTION ===
\section{Program Funding and Execution}
\minibib
\subsection{BA, commitment, obligation, expenditure, outlay}
\begin{enumerate}
  \item \emph{BA} available (legal authority).
  \item \emph{Commitment} (administrative reservation).
  \item \emph{Obligation} (legal liability via contract/order).
  \item \emph{Expenditure} (payment recorded).
  \item \emph{Outlay} (cash disbursed/Treasury).
  \item[] Source references~\autocite{DoDFMR-Vol3,Treasury-GInvoicing}.
\end{enumerate}

\subsection{Appropriation categories, scope, obligation windows}
\Tab{\ref{tab:appropriation_categories}} shows the Colors of Money with their spending categories and obligation windows.  Figure~\ref{fig:ppbe_approps_lifecycle} from \S~\ref{sec:colors_of_money} visually shows when each category is Current, Expired, or Cancelled.

\begin{center}
  \begin{talltblr}[
      note{a} = {\footnotesize \ac{scn} is a specific procurement with a 5-year obligation window.},
      label   = {tab:appropriation_categories},
      caption = {Appropriation categories, scope, obligation windows},
      entry   = {Colors of Money},
      remark{Source} = {\tabCite{DoDFMR-Vol3}}
    ]
    {colspec = {@{} F L L @{}}}
    \toprule
    {Category}      & {Typical Scope}                                        & {Obligation Window} \\
    \midrule
    RDT\&E          & Science \& technology; development; test \& evaluation & 2 years             \\
    OPN\TblrNote{a} & End items/modernization and spares                     & 3 years             \\
    OMN             & Operations, maintenance, training, minor mods          & 1 year              \\
    MILCON          & Facilities construction and real property              & 5 years             \\
    MILPERS         & Personnel pay                                          & 1 year              \\
    \bottomrule
  \end{talltblr}
\end{center}

\subsection{Funding policies}
\begin{description}
  \item[\textbf{Annual. (\ac{omn}/\ac{milpers}).}] Fund only bona fide needs of the fiscal year.
  \item[\textbf{Incremental. (\ac{rdte}).}] Fund work as it occurs by fiscal year.
  \item[\textbf{Full-funding. (Procurement/\ac{milcon}).}] Fund the total usable end-item at award.\footnote{Exceptions to Full funding are select \ac{scn} programs (CVNs, SSBNs, and LHD/LHAs) with Congressional approval.}
\end{description}

\emph{Common techniques/exceptions:} Advance Procurement (long-lead items), Multiyear Procurement, Economic Order Quantity, Cost-to-Complete when provided in law.

\subsubsection{Exceptions to the Full Funding Policy}
\label{sec:full-funding-exceptions}
Below is the list of the key exceptions to the Full Funding Policy, what they allow, and the must-know rules/approvals (Table~\ref{tab:full_funding_exceptions} shows full details).

\begin{center}
  \begin{talltblr}[
      caption = {Key exceptions to the Full Funding Policy: what they allow and what to remember},
      entry   = {Exceptions to Full Funding},
      label   = {tab:full_funding_exceptions},
      remark{\tNote} = {\\%
          (i) \ac{eoq} purchases are commonly approved within \ac{myp} or other explicit multi-year constructs;\\%
          (ii) Block Buy resembles \ac{myp} in intent (savings) but uses program-specific authorization;\\%
          (iii) Plan for cancellation liability under \ac{myp} when required.},%
      remark{\tSource} = {\tabCite{USC-3501,DFARS217-172,DoDFMR-Vol3}}
    ]
    {
      colspec = {@{} Q[wd=.15\linewidth,cmd=\merriweatherblack\RaggedRight\small] P{.15\linewidth} L L @{}}
    }
    \toprule
    {Exception}                                    & {Typical Appropriation}                   & {What it allows}                                                                                                                                                    & {Must-know rules / approvals}                                                                                                                                                          \\
    \midrule
    MYP                                      & APN, OPN, SCN              & Single contract covering multiple fiscal years of buys to achieve predictable demand and unit-cost savings; may include EOQ purchases common across years.     & Requires specific congressional authorization; stable requirements/design; credible savings; low technical risk; term generally up to 5 years; plan/budget any cancellation liability. \\
    Advance Procurement (incl. LLTM/EOQ) & APN, OPN, SCN              & Limited \emph{early} funding (often 1--2 years before the main ``buy'' year) for long-lead components or EOQ lots to protect schedule or achieve price breaks. & Narrow and part-specific; must be justified in budget docs; EOQ commonly tied to MYP (or explicit block authority); does \emph{not} equal full funding of the end item.      \\
    BB                                        & Primarily SCN                        & A single contract buying multiple ships (or end items) across fiscal years to gain tooling/learning-curve efficiencies.                                             & Requires explicit congressional authorization; funds still obligated year-by-year; not under the MYP statute, so oversight mechanics differ.                                      \\
    CTC                                       & SCN                                  & Additional funds in a later FY to finish an item (e.g., ship) when matured estimates exceed prior appropriations.                                                   & Completes original approved scope (not added capability); appears as a distinct request/justification; common in long-duration shipbuilding.                                           \\
    RCOH for CVN                         & SCN (with prior Advance Procurement) & Mid-life refueling and major overhaul funded incrementally over multiple years; early procurement of nuclear fuel/critical components.                              & Recognized exception due to size/complexity; incremental SCN with advance procurement years ahead of the principal execution.                                                     \\
    \bottomrule
  \end{talltblr}
\end{center}

\subsection{Navy Sustainment Funding: TYCOM v.s. Fleet (O\&S)}
\begin{description}
  \item[\textbf{Who holds O\&M,N.}] Type Commanders (\emph{TYCOMs}; e.g., COMNAVSURFOR, COMNAVAIRFOR, COMSUBFOR) are the primary executors of \ac{omn} for unit readiness: flying hours, steaming days, intermediate-level maintenance, training, and afloat/ashore support.
  \item[\textbf{Fleet Commanders.}] PACFLT and USFF set operational priorities and readiness goals, adjudicate contingencies, and may centrally manage certain Fleet-wide initiatives; they influence TYCOM allocations but most execution occurs at the TYCOM.
  \item[\textbf{Program Office (SYSCOM/PEO).}] Funds in-service engineering, tech data, Diminishing Manufacturing Sources, and modernization using \ac{opn}/\ac{apn} (end items/kits) and \ac{rdte}/\ac{omn} (installation, trials) as appropriate; manages sustainment \acp{idiq}/PBL with color-of-money compliance.
  \item[\textbf{Who pays what.}] \ac{opn}/\ac{apn}: hardware/software upgrade kits, initial spares; \ac{omn}: installs, fleet introduction/training, minor mods, repair parts consumption; \ac{rdte}: development/test of engineering changes. NAVSUP/NWCF activities recover costs via rates for supply/repair.
  \item[\textbf{Board cue.}] In O\&S, TYCOMs execute most \ac{omn}; Program Managers fund modernization and in-service support from \ac{opn}/\ac{apn}/\ac{rdte}. Align funding lines with the \ac{wbs} and the Colors-of-Money rules to avoid Anti-Deficiency Act risk.
\end{description}

% Optional quick bullets (kept minimal and consistent with your style)
Quick information regarding exceptions to the full funding policy:
\begin{description}
  \item[\textbf{\ac{myp}.}] Up to 5 years; stable design; verifiable savings; cancellation liability planned.
  \item[\textbf{Advance Procurement (\ac{lltm}/{\ac{eoq}}).}] Focused early buys for schedule/savings; not full funding.
  \item[\textbf{\ac{bb}.}] Congressionally authorized multi-ship buys; obligate by \ac{fy}; not the \ac{myp} statute.
  \item[\textbf{\ac{ctc}.}] Complete original scope when estimates mature upward; separate \ac{scn} request.
  \item[\textbf{\ac{rcoh}.}] Planned mid-life overhaul; incremental \ac{scn} with advance procurement of cores/critical material.
\end{description}

\subsection{Which money when (by acquisition phase)}
Table~\ref{tab:msa_phase_approp} shows the nominal types of appropriation used for \ac{msa}, \ac{tmrr}, \ac{emd}, \ac{pd.phase}, and \ac{os.phase}

\begin{center}\begin{talltblr}[
    caption = {Appropriation types per phase},
    entry   = {Appropriations per Phase},
    label   = {tab:msa_phase_approp},
  ]
  {
    colspec = {F L},
    stretch = 1.2 % (Optional) avoid row “stretching”; see notes below
  }
  \toprule
  {Phase} & {Typical appropriation(s)}\\
  \midrule
  MSA     & RDT\&E (analyses, prototyping); small OMN for studies\\
  TMRR    & RDT\&E (risk reduction, development, test)\\
  EMD     & RDT\&E (develop/build/test/qualify); selected long-lead/AP/EOQ in Procurement when authorized\\
  PD      & Procurement (e.g., OPN); limited RDT\&E for fixes\\
  OS      & OMN sustainment; spares via Procurement as applicable\\
  \bottomrule
\end{talltblr}\end{center}

\subsection{MILCON v.s.\ O\&M sustainment v.s.\ OPN modernization}
\emph{Rule of thumb:} Real property construction $\Rightarrow$ \ac{milcon}; keeping systems running (repair/overhaul, services) $\Rightarrow$ \ac{omn}; new capability/end-item upgrades $\Rightarrow$ Procurement (e.g., \ac{opn}).

\subsection{Field Activity Financial Management (NWCF quick hits)}
\begin{description}
  \item[\textbf{Goal.}] \ac{nwcf} activities price to \emph{break even over time}; in-year result is \textbf{\ac{nor}}, cumulative is \textbf{\ac{aor}}.
  \item[\textbf{Rates (\ac{slr}).}] Recover direct labor + overhead + G\&A; shocks flow to NOR/AOR and are managed over time.
  \item[\textbf{Carryover.}] Unfilled orders at \ac{fy} end; subject to limits/management.
  \item[\textbf{\ac{igt} via G-Invoicing.}] Treasury-mandated platform for intra-gov orders/settlement; apportionment/availability rules still govern customer funds.
\end{description}
%====================
% End of file
%====================
\ifSubfilesClassLoaded{
  \RenewDocumentCommand{\entryname}{}{\textbf{\color{Modern} Acronym}}
  \RenewDocumentCommand{\descriptionname}{}{\textbf{\color{Modern} Definition}}
  \printnoidxglossary[
    type=\acronymtype,
    title=Acronyms,
    style=long-booktabs
]}{}
\end{document}
