% !TEX root = ../main.tex
% !TEX program = lualatex
\documentclass[../main.tex]{subfiles}
\IfSubfilesClassLoaded{\externaldocument{\subfix{../build/main}}}{}
% =============================================================================
% APPENDIX: KEY ROLES IN ACQUISITION AND MODERNIZATION
%
% DESCRIPTION: This appendix provides a comprehensive list of key roles and
%              their responsibilities within the acquisition and modernization
%              enterprise.
% =============================================================================
\begin{document}
\ifSubfilesClassLoaded{\chapter{EDO Study Guide}}{}
%====================
% Contents here
%====================

%====================
% Contents
%====================
% === ROLES ===
\section{Key Roles in Acquisition and Modernization}
% Program Authority (PA), Technical Authority (TA), and Contracting authority remain distinct.
\begin{description}[style=nextline, labelsep=0.5em, font=\bfseries]
	\item[\textbf{Department acquisition governance.}]
	Sources: SECNAVINST~5400.15D; SECNAVINST~5000.2G; DoDI~5000.85.
      \begin{itemize}
        \item \ac{asnrdanda}, the \ac{sae} and \ac{cae} for the \ac{don}, sets acquisition policy, charters \acp{peo}/\acp{drpm}, and adjudicates milestone decisions when delegated by the Milestone Decision Authority.
        \item \acp{peo} and \acp{drpm} hold delegated program authority, establish baselines, and are accountable for translating capability needs into executable acquisition strategies.
        \item \ac{syscom} Commanders provide the workforce, infrastructure, and \ac{ta} warrants that underpin \ac{peo} execution inside the Navy matrix construct.
      \end{itemize}

	\item[\textbf{Program Authority chain.}]
	Sources: SECNAVINST~5000.2G; DoDI~5000.85.
      \begin{itemize}
        \item \textbf{\ac{peo}/\ac{pa}}: Owns program outcomes, charters \acp{pm}, approves acquisition strategies and baselines.
        \item \textbf{\ac{drpm}}: Direct-report program leads for special access or priority portfolios with the same milestone authority as \acp{peo}.
        \item \textbf{\ac{spm}/\ac{pm}}: Accountable for cost/schedule/performance, leads risk management, orchestrates \acp{ipt}.
        \item \textbf{\ac{shapm}}: Delivers new hulls; coordinates design/build/activation sequencing.
        \item \textbf{\ac{slm}}: Drives sustainment and modernization packages for in-service assets.
        \item \textbf{\ac{parm}/\ac{spd}}: Sponsors platform-level upgrades and alteration packages.
      \end{itemize}

	\item[\textbf{Technical Authority chain.}]
	Source: SECNAVINST~5400.15D.
      \begin{itemize}
        \item \textbf{\ac{uta} / Chief Engineer}: Issues enterprise technical policy, owns warrants.
        \item \textbf{\acp{twh}}: Certifies design compliance, adjudicates departures/concurrence packages.
        \item \textbf{Engineering Agents (\ac{isea}, \ac{da}, \ac{aea}, \ac{sia}, \ac{tda})}: Execute lifecycle engineering; provide in-service engineering decisions under delegated authority.
      \end{itemize}

	\item[\textbf{NAVSEA headquarters roles.}]
	Source: EDO Coursebook Module~2.1.2 (NAVSEA Organization, 2025 edition).
      \begin{itemize}
        \item \textbf{SEA~01 Comptroller}: \ac{bso} lead for \ac{navsea} financial governance and funds control.
        \item \textbf{SEA~02 Contracts}: Enterprise contracting authority, policy, and warrant management.
        \item \textbf{SEA~03 Cyber Engineering and Digital}: Drives cyber resiliency, digital engineering, and data transformation initiatives.
        \item \textbf{SEA~04 Industrial Operations}: Oversees public shipyards, maintenance execution, and quality assurance.
        \item \textbf{SEA~05 Chief Engineer}: Chief \ac{ta}; issues warrants, certifies designs, and maintains technical standards.
        \item \textbf{SEA~06 Sustainment}: Manages product support strategies and lifecycle logistics integration.
        \item \textbf{SEA~07 Undersea Warfare}: Dual-hatted as \ac{peouws}; oversees undersea combat systems sustainment.
        \item \textbf{SEA~08 Nuclear Propulsion}: Three-hatted nuclear propulsion authority across \ac{don} and \ac{doe} roles.
        \item \textbf{SEA~09 Safety and Regulatory Compliance}: Aligns enterprise safety governance and reporting.
        \item \textbf{SEA~10 Total Force and Corporate Ops}: Manages workforce planning, corporate services, and governance.
        \item \textbf{SEA~21 In-Service Ships/\ac{cnrmc}}: Leads surface-ship sustainment and modernization (PMS~321/326/339/421/443/451, SEA~21I, \ac{surfmepp}).
      \end{itemize}

	\item[\textbf{NAVWAR enterprise directorates.}]
	Source: EDO Coursebook Module~2.1.3 (NAVWAR Enterprise, 2024 edition).
      \begin{itemize}
        \item \textbf{Code~1.0 Comptroller}: Budget formulation/execution, \ac{bso} duties, and funds certification.
        \item \textbf{2.0 Contracts}: Contracting policy, strategy reviews, and award/administration oversight.
        \item \textbf{3.0 Counsel}: Acquisition law, ethics, protests, and claim resolution.
        \item \textbf{4.0 Logistics and Fleet Support}: Lifecycle logistics, technical data, and fleet distance support integration.
        \item \textbf{5.0 Chief Engineer/\ac{ta}}: Enterprise architectures, interoperability, and cyber certification authority.
        \item \textbf{6.0 Program Management}: Portfolio governance, milestone preparation, and \ac{peo} integration.
        \item \textbf{7.0 Science and Technology}: S\&T portfolio management, prototyping, and technology transition.
        \item \textbf{8.0 Corporate Operations}: Workforce, \ac{cio} services, facilities, security, and public affairs.
        \item \textbf{FRD-100 Fleet Support}: Deployed engineering assistance and sustainment response.
        \item \textbf{FRD-200 Installations}: Shore/afloat \ac{c4i} installation planning and cutover execution.
      \end{itemize}

	\item[\textbf{Program Executive Offices (Navy portfolios).}]
	Sources: SECNAVINST~5400.15D; EDO Coursebook Modules~2.1.4 and~3.1.5.
      \begin{itemize}
        \item \textbf{\ac{peocvn}}: Designs, builds, and sustains nuclear-powered aircraft carriers (e.g., PMS~312/378/379).
        \item \textbf{\ac{peoiws}}: Develops and sustains ship/submarine combat systems; mission-aligned IWS directorates cover sensors, weapons, C2, and allied integration.
        \item \textbf{\ac{peoships}}: Oversees surface combatant and amphibious ship construction and modernization.
        \item \textbf{\ac{peousc}}: Leads LCS, FFG~62, expeditionary, and unmanned surface/undersea portfolios.
        \item \textbf{Team Submarines (\ac{peossn}, \ac{peossbn}, \ac{peocolumbia}, \ac{peouws})}: Manages attack/strategic submarine acquisition, in-service support, and undersea combat systems.
        \item \textbf{\ac{peoc4i}}: Delivers fleet \ac{c4i}; PMW~1XX focus on capability development, PMW~7XX on platform integration.
        \item \textbf{\ac{peodigital}}: Provides enterprise digital services (e.g., Flank Speed) across the \ac{don}.
        \item \textbf{\ac{peomlb}}: Modernizes manpower, logistics, and business IT systems.
      \end{itemize}

	\item[\textbf{Contracting authority (\ac{ko} family).}]
	Sources: FAR; DFARS; NMCARS; EDO Coursebook Module~3.2.1.
      \begin{itemize}
        \item \textbf{\ac{pco}}: Plans the acquisition, synchs with \acp{pm} before solicitation, awards and signs contracts/mods.
        \item \textbf{\ac{aco}}: Oversees post-award performance, surveillance, and payment/contract administration (often \ac{dcma}/\ac{navsea} field activities).
        \item \textbf{\ac{tco}}: Leads partial/full terminations, settlement negotiations, and equitable adjustments.
        \item \textbf{\ac{ko} warrant}: Defines the dollar/authority limits; only the warranted \ac{ko} can bind or obligate the Government.
        \item \textbf{\ac{cor} / assistant \ac{pm} / engineering support}: Provide technical surveillance and acceptance recommendations; cannot direct work or obligate funds (FAR Parts~1 and~42).
        \item \textbf{\ac{hca}}: Approves actions such as letter contracts and high-value single-award \acp{idiq}; may delegate no lower than flag/Senior Executive levels within the \ac{don} per FAR~1.601 and DFARS/NMCARS supplements.
        \item \textbf{\ac{sae} (\ac{asnrdanda})}: Signs D\&Fs for multiyear contracting, extraordinary relief, and other actions reserved to the Service Acquisition Executive under FAR Subparts~1.7 and~17.1.
      \end{itemize}

	\item[\textbf{Program Manager / \ac{ko} partnership.}]
	Sources: DoDI~5000.85; EDO Coursebook Module~3.2.1.
      \begin{itemize}
        \item \ac{pm} integrates warfighter need, technical baseline, and budget; \ac{ko} ensures statutory/regulatory compliance and contract enforceability.
        \item Both align on acquisition strategy, competition approach (\ac{cica}), incentives, and change management before \ac{rfp} release or modification execution.
      \end{itemize}

	\item[\textbf{Source selection governance.}]
	Sources: FAR Parts~1, 5, and~15; NAVSEA Source Selection Guide (2022); EDO Coursebook Module~3.2.2.
      \begin{itemize}
        \item \textbf{\ac{ssa}}: Senior official who approves the Source Selection Plan, receives \ac{ssac}/\ac{sseb} recommendations, and signs the best-value decision memorandum.
        \item \textbf{\ac{ssac}}: Advisory council that synthesizes \ac{sseb} findings, compares proposals across factors, and briefs the \ac{ssa} on trade-offs.
        \item \textbf{\ac{sseb}}: Multi-disciplinary evaluators (technical, management, past performance, cost/price) who rate proposals against Section~M factors.
        \item \textbf{Small Business Professional / Competition Advocate}: Confirms set-aside decisions, reviews subcontracting plans, and endorses synopsis waivers.
        \item \textbf{Cost/Price Analyst \& Legal Counsel}: Validate reasonableness determinations, alignment of Sections~L/M, and clause sufficiency before release.
        \item \textbf{Award Fee Determining Official \& Award Fee Board}: Plans award-fee periods, chairs performance reviews, and signs the determination memo authorizing or withholding fee payments (FAR Part~16; NAVSEA Source Selection Guide).
      \end{itemize}

	\item[\textbf{Fiscal control \& certification.}]
	Source: DoD FMR~7000.14-R Volume~3.
      \begin{itemize}
        \item \textbf{Funds Certifying Official/Comptroller}: Verifies purpose/time/amount before obligation; \ac{ada} safeguard.
        \item \textbf{Program/Budget Analyst}: Tracks execution, monitors reprogramming thresholds, prepares obligation/expenditure burn-down.
        \item \textbf{Resource allocation chain}: OUSD(C) apportions to the Navy; OPNAV (N82/FMB) allocates to \acp{bso}; \ac{syscom} comptrollers (e.g., SEA~01) issue suballocations to executing activities.
      \end{itemize}

	\item[\textbf{PPBE resource sponsors.}]
	Source: EDO Coursebook Module~3.1.1 (PPBE, 2025 edition).
      \begin{itemize}
        \item \textbf{\ac{cape}}: Provides independent cost assessments and program evaluation across the Department.
        \item \textbf{\ac{n8}}: Builds the Navy \ac{pom}, balancing capability and fiscal constraints across the \ac{fydp}.
        \item \textbf{\ac{n9}}: Integrates warfare requirements and advocates mission-area investments.
        \item \textbf{\ac{n91}}: Conducts cross-domain mission integration and orchestrates \ac{pom} issue resolution.
      \end{itemize}

	\item[\textbf{Congressional resource chain.}]
	Source: EDO Coursebook Module~3.1.2 (Congressional Enactment, 2025 edition).
      \begin{itemize}
        \item \textbf{\ac{hasc} / \ac{sasc}}: Authorize defense programs and policy in the annual \ac{ndaa}.
        \item \textbf{\ac{hac} / \ac{sac}}: Produce appropriations bills that provide \ac{ba} to execute Navy programs.
        \item \textbf{\ac{hbc} / \ac{sbc}}: Set topline guidance and 302 allocations through the budget resolution process.
        \item \textbf{\ac{cbo}, \ac{crs}, \ac{gao}}: Supply independent scoring, research, and oversight that shape committee deliberations.
      \end{itemize}

	\item[\textbf{Cyber/Authorizations (when IT/C2 in scope).}]
	Source: SECNAVINST~5000.2G.
      \begin{itemize}
        \item \textbf{\ac{ao}}: Grants Authority to Operate; enforces \ac{rmf} controls that drive design and integration requirements.
      \end{itemize}
\end{description}
%====================
% End of file
%====================
\ifSubfilesClassLoaded{
  \renewcommand*{\entryname}{\textbf{\color{Modern} Acronym}}
  \renewcommand*{\descriptionname}{\textbf{\color{Modern} Definition}}
  \printnoidxglossary[
    type=\acronymtype,
    title=Acronyms,
    style=long-booktabs
]}{}
\end{document}

