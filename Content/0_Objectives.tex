% !TEX root = ../main.tex
% !TEX program = lualatex
\documentclass[../main.tex]{subfiles}
\IfSubfilesClassLoaded{\externaldocument{\subfix{../build/main}}}{}
% =============================================================================
% CHAPTER 0: MUST-KNOW OBJECTIVES
%
% DESCRIPTION: This chapter lists the must-know objectives for the EDO board
%              preparation, covering a wide range of topics from command
%              chains to contracting fundamentals.
% =============================================================================
\begin{document}
\ifSubfilesClassLoaded{\chapter{EDO Study Guide}}{}
%====================
% Contents here
%====================

%====================
% Contents
%====================
% === Objectives section (kept as-is) ===
\section{Must-Know Objectives}
By the end of prep, you can rapidly:
\begin{enumerate}
	\item Explain the \ac{ops-chain} per Title 10 (President $\to$ \ac{secdef} $\to$ \ac{ccdr}).
	\item Describe how acquisition governance is distinct from the operations chain and map \ac{pa} v.s.\ \ac{ta} decision rights.
	\item Sketch how \ac{navsea} is organized (headquarters codes and \acp{nwc}); state what \acp{syscom} do v.s.\ \acp{peo} and warfare centers.
	\item Identify \ac{navwar} mission lines (\ac{niwclant}, \ac{niwcpac}, and \ac{nsfa}) and what \acp{frd} deliver to the fleet.
	\item List Navy \acp{peo} and their portfolio focus areas.
	\item Recognize current \ac{edo} flag billets and associated enterprise responsibilities.
	\item Describe the \ac{ta} chain (\ac{uta}, \ac{twh}, engineering agents), how warrants are delegated, and how departures are adjudicated.
	\item Explain how \ac{pa}, \ac{ta}, and the \ac{ko} collaborate to balance warfighter needs, technical rigor, and statutory compliance.
	\item Distinguish \ac{nwcf} v.s.\ mission-funded execution, including stabilized labor rates, \ac{slr}, \ac{nor}, and \ac{aor}.
	\item Draw and discuss \ac{ppbe} phases, key players, major work products, and the annual timeline.
	\item Explain \ac{py}, \ac{cy}, \ac{by}, and the \ac{pom} across the \ac{fydp} and how they drive submissions.
	\item Distinguish Authorization v.s.\ Appropriation, identify who drafts each bill, and recall the 12 regular appropriations titles.
	\item Sequence \ac{ba}, commitment, obligation, expenditure, and outlay; map colors of money to obligation and expense windows.
	\item Explain annual, incremental, and full funding policies plus the limited exceptions (e.g., advance procurement, multiyear contracting) and when they apply.
	\item Select the correct reprogramming vehicle (\ac{btr}, \ac{ir}, \ac{pa-reprog}, \ac{lt}) based on dollar/value thresholds and statutory triggers.
	\item Articulate why the Government uses contracts and list the four elements required for a binding instrument (mutual assent, consideration, capacity, lawful purpose).
	\item Describe the Government-Contractor relationship (transactional, professional, collaborative, constrained) and the ethics guardrails that shape interactions.
	\item Summarize the \ac{far} system hierarchy (\ac{far}, \ac{dfars}, \ac{nmcars}, \ac{supship} Operations Manual, \ac{navsea} Source Selection Guide, and \ac{navsea} Contracts Handbook)~\autocite{FAR,DFARS,NMCARS,SOM-Ch3-2023,NAVSEA-Source-Selection-Guide-2022,NAVSEA-Contracts-Handbook-2023}.
	\item List the three types of contracting officers (\ac{pco}, \ac{aco}, \ac{tco}) and the scope of authority each exercises under their warrant.
	\item Compare \ac{pm} and \ac{ko} responsibilities across the acquisition lifecycle and explain how the partnership balances program execution with procurement law.
	\item Describe \ac{cica} requirements and recall the seven exceptions in \autocite[\S~6.302]{FAR} with example use cases.
	\item Differentiate responsiveness (sealed bidding) v.s.\ responsibility (contractor qualifications) and state when each determination is required.
	\item Distinguish a Justification \& Approval from a Determination \& Findings, including content and approval thresholds.
	\item Recognize why the Department of Defense relies on contracts in acquisition.
	\item Recognize the legal nature of a contract as a binding agreement.
	\item Recognize the \ac{far} and its supplements as the governing framework for procurement.
	\item Recognize the three types of contracting officers.
	\item Identify how contracting officers and program managers complement one another across the acquisition lifecycle.
	\item Recognize key differences in background, rules, and responsibilities between a \ac{pm} and a \ac{ko}.
	\item Identify the statutory and regulatory requirements associated with pursuing competition.
	\item Identify critical contracting terms such as responsiveness, responsibility, Justification and Approval (J\&A), and Determination and Findings (D\&F).
	\item Outline the Uniform Contract Format (Sections~A through M), flag which sections must be tailored for Navy solicitations, and align them to owning team members.
	\item State the minimum documentation required before release (approved acquisition plan/strategy, funds certification, synopsis timing, legal review) and how updates are controlled through amendments.
	\item Explain how Section~L instructions must align with Section~M evaluation factors, including basis of award (\ac{lpta} v.s.\ tradeoff) and relative importance statements.
	\item Identify the solicitation preparation roles of the \ac{ko}, \ac{pm}, Small Business Professional, Competition Advocate, legal counsel, cost/price analysts, and the \ac{sseb}/\ac{ssac}/\ac{ssa} triad.
	\item Describe permitted industry communications (\acp{rfi}, draft \acp{rfp}, Q\&A, site visits) versus prohibited conduct after release, and the timelines for sam.gov notices and proposal preparation.
	\item Compare sealed bidding and negotiated procurement, identifying when each method applies and how discussions, price competition, and responsiveness differ.
	\item Differentiate fixed-price versus cost-reimbursement contract families (\ac{ffp}, \ac{fpif}, \ac{fpepa}, \ac{fpr}, \ac{cpff}, \ac{cpif}, \ac{cpaf}), explain who bears cost risk, and map typical usage to Materiel Solution Analysis, \ac{tmrr}, \ac{emd}, Production/Deployment, and O\&S phases.
	\item Distinguish incentive fees from award fees and describe how award-fee boards determine ratings and authorize payment.
	\item Calculate a fixed-price incentive (firm target) outcome using target cost, target profit, share ratios, and ceiling price, and determine the point of total assumption.
	\item Characterize \ac{idiq} contracts, time-and-materials awards, labor-hour vehicles, and undefinitized contract actions (\ac{uca}) and explain when each supports Navy solicitations.
	\item Explain the simplified acquisition threshold (\ac{sat}), the benefits of using simplified acquisition procedures (\ac{sap}), and the associated small-business reservation requirements.
	\item Summarize Small Business Administration socio-economic programs (8(a), \ac{hubzone}, \ac{sdvosb}, \ac{wosb}) and how they drive set-aside or sole-source decisions.
	\item Explain Other Transaction agreements (\ac{ot}) and Small Business Innovation Research (\ac{sbir}) phases and protections.
	\item Describe the major technical data rights categories (unlimited, government purpose, limited/restricted) and the constraints each imposes.
	\item Identify the critical Uniform Contract Format sections that must stay synchronized across the contract and solicitation.
\end{enumerate}
% End of file
%====================
\ifSubfilesClassLoaded{
  \renewcommand*{\entryname}{\textbf{\color{Modern} Acronym}}
  \renewcommand*{\descriptionname}{\textbf{\color{Modern} Definition}}
  \printnoidxglossary[
    type=\acronymtype,
    title=Acronyms,
    style=long-booktabs
]}{}
\end{document}

