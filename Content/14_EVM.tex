% !TEX root = ../main.tex
% !TEX program = lualatex
\documentclass[../main.tex]{subfiles}
\IfSubfilesClassLoaded{\externaldocument{\subfix{../build/main}}}{}
%====================
% File: SECTION_FILE_NAME.tex
%====================
\begin{document}
\ifSubfilesClassLoaded{\chapter{EDO Study Guide}}{}
% The section title is not used for regular content. The title is set in the main document.
%====================
% Contents
%====================
% Enter content here.
% === EVM (EDO 3.6.1 & 3.6.2) ===
\section{Earned Value Management (EVM)}
\minibib

\subsection{What EVM is}

\ac{evm} is a project management technique that integrates scope, schedule, and cost to assess project performance and progress. It provides a quantitative measure of project performance by comparing the planned work with the actual work completed and the associated costs.
\ac{evm} is widely used in government and defense projects to ensure that projects are delivered on time and within budget.

\subsection{Key Components of EVM}
\srcCite{edo-3-6-1-evm-intro-2025, edo-3-6-2-evm-analysis-2025}
\begin{description}
	\item [\textbf{\ac{pv} / \ac{bcws}}]: The budgeted cost for the work scheduled to be completed by a specific date.
	\item [\textbf{\ac{ev} / \ac{bcwp}}]: The budgeted cost for the work actually completed by a specific date.
	\item [\textbf{\ac{ac} / \ac{acwp}}]: The actual cost incurred for the work completed by a specific date.
	\item [\textbf{\ac{bac}}]: The total budget allocated for the project.
	\item [\textbf{\ac{eac}}]: The forecasted total cost of the project based on current performance.
	\item [\textbf{\ac{cpi}}]: A measure of cost efficiency, calculated as $\text{CPI} = \text{EV} / \text{AC}$. A \ac{cpi} less than 1 indicates a cost overrun.
	\item [\textbf{\ac{spi}}]: A measure of schedule efficiency, calculated as $\text{SPI} = \text{EV} / \text{PV}$. An \ac{spi} less than 1 indicates a schedule delay.
	\item [\textbf{\ac{sv}}]: The difference between the earned value and the planned value, calculated as $\text{SV} = \text{EV} - \text{PV}$. A negative \ac{sv} indicates a schedule delay.
	\item [\textbf{\ac{cv}}]: The difference between the earned value and the actual cost, calculated as $\text{CV} = \text{EV} - \text{AC}$. A negative \ac{cv} indicates a cost overrun.
\end{description}
\Info{The first three are equivalent terms and should know both.  This guide will use the first set (\ac{pv}, \ac{ev}, \ac{ac})}

\subsection{When to Use EVM}
\ac{evm} is mandated on cost or incentive contracts (and applicable subcontracts) when the program meets the policy thresholds and the effort is discretely measurable~\autocite{edo-3-6-1-evm-intro-2025,DFARS}. Key decision points for the \ac{pm} include:
\begin{description}
	\item[\$100 million and above:] Implement the full ANSI/EIA-748 standard and ensure the contractor’s system is formally validated by \ac{dcma}.
	\item[\$20 million to \$99.9 million:] Implement ANSI/EIA-748 guidelines, with validation required when performance indicates risk.
\end{description}
Additional policy considerations~\autocite{edo-3-6-1-evm-intro-2025,DoDI5000-85}:
\begin{itemize}
	\item Applicability determinations must confirm the work scope can be discretely measured before mandating \ac{evm}.
	\item The Milestone Decision Authority may approve \ac{evm} on \ac{ffp}, time-and-materials, or \ac{loe} contracts, but such use is discouraged absent clear benefit.
	\item For efforts below \$20 million, the \ac{pm} conducts a risk-based cost-benefit analysis to decide whether \ac{evm} adds value.
	\item Contracts that require \ac{evm} must also deliver \ac{ipmdar} data and complete an \ac{ibr} within six months of award.
\end{itemize}
\subsection{EVM Compliance}
\ac{evm} clauses are flowed in the solicitation and award package (DFARS 252.234-7001 and 252.234-7002) alongside the Contractor Business Systems clause (DFARS 252.242-7005) to anchor validation and surveillance requirements~\autocite{DFARS}. Compliance expectations~\autocite{edo-3-6-1-evm-intro-2025,DoDI5000-85}:
\begin{itemize}
	\item The program office integrates \ac{evm} planning into the \ac{wbs}, maintains the Performance Measurement Baseline, and ensures timely receipt and analysis of \ac{ipmdar} submissions.
	\item \ac{dcma} leads \ac{evm} system acceptance, validation, and ongoing surveillance against ANSI/EIA-748 and the DoD Earned Value Management Implementation Guide; \ac{dcaa} supports with accounting system audits.
	\item Service focal points such as Acquisition Data and Analytics and the \ac{don} Center for \ac{evm} adjudicate applicability determinations and coordinate policy updates.
	\item \ac{supship} can execute validation and surveillance responsibilities for shipbuilding programs on behalf of \ac{dcma}.
\end{itemize}

\subsection{EVM Compliance}
\ac{evm} clauses are flowed in the solicitation and award package (DFARS 252.234-7001 and 252.234-7002) alongside the Contractor Business Systems clause (DFARS 252.242-7005) to anchor validation and surveillance requirements~\autocite{DFARS}. Compliance expectations~\autocite{edo-3-6-1-evm-intro-2025,DoDI5000-85}:
\begin{itemize}
	\item The program office integrates \ac{evm} planning into the \ac{wbs}, maintains the Performance Measurement Baseline, and ensures timely receipt and analysis of \ac{ipmdar} submissions.
	\item \ac{dcma} leads \ac{evm} system acceptance, validation, and ongoing surveillance against ANSI/EIA-748 and the DoD Earned Value Management Implementation Guide; \ac{dcaa} supports with accounting system audits.
	\item Service focal points such as Acquisition Data and Analytics and the \ac{don} Center for \ac{evm} adjudicate applicability determinations and coordinate policy updates.
	\item \ac{supship} can execute validation and surveillance responsibilities for shipbuilding programs on behalf of \ac{dcma}.
\end{itemize}

\subsection{EVM Principal Players}
\srcCite{edo-3-6-1-evm-intro-2025}
\begin{description}
	\item[\textbf{\ac{pmo}}:] Implements \ac{evm} on the contract, ensures solicitations and awards contain the required clauses, and works with the contracting activity to tailor reporting and \ac{ims} requirements.
	\item[\textbf{\ac{dcma}}:] Serves as the DoD \ac{evm} Executive Agent by validating, accepting, and surveilling contractor systems, maintaining the official acceptance roster, and analyzing \ac{ipmdar} and schedule submissions (including the 14-point \ac{ims} review) to focus government attention on emerging issues.
	\item[\textbf{\ac{dcaa}}:] Conducts accounting system audits, corroborates cost data, and supports \ac{dcma} during surveillance events and reviews.
	\item[\textbf{\ac{ada.acq}}:] Acts as the department-wide focal point for \ac{evm} policy, guidance, and competency management; issues interpretations to maintain consistent application across programs.
	\item[\textbf{\ac{don} Center for \ac{evm}}:] Coordinates applicability determinations with the Deputy Assistant Secretary of the Navy for Acquisition Policy and Budget and the Office of the Secretary of Defense, and serves as the Navy’s central point of contact for \ac{evm} matters.
	\item[\textbf{\ac{supship}}:] Performs many \ac{dcma}/\ac{dcaa} surveillance roles for shipbuilding contracts when delegated by the cognizant program office.
\end{description}

\subsection{PMB}

\Fig{\ref{fig:pmb}} shows the characteristic S-curve depicting cumulative $\text{PV}/\text{BCWS}$, while \Fig{\ref{fig:pmb_dev}} summarizes the progression from defining the \ac{sow} to final \ac{pmb} adjustments. \ac{pmb}s are:
\begin{itemize}
	\item Scoped, scheduled, and budgeted plans for the authorized work
	\item Time-phased budgets that align to the master schedule
	\item The basis for cost and schedule performance management
	\item Effectively the \ac{pv} for the entire project
\end{itemize}

\begin{figure}
	\centering
	\includegraphics[width=0.5\linewidth]{PMB.png}
	\caption[PMB ``S''-Curve]{PMB Cost v.s.\ Time chart showing the ``S''-curve. \srcCite{edo-3-6-1-evm-intro-2025}.}
	\label{fig:pmb}
\end{figure}

\begin{figure}
	\centering
	\includegraphics[width=0.85\linewidth]{pmb_development.png}
	\caption[PMB Development Flow]{The development flow for PMB. \srcCite{edo-3-6-1-evm-intro-2025}.}
	\label{fig:pmb_dev}
\end{figure}

\subsubsection{PMB Development}
\begin{enumerate}
	\item[STEP 1:] \emph{Define all work}. Decompose the effort using the \ac{wbs} and align it with the organizational structure. Control accounts are the natural management points where \acp{cam} are assigned, and each control account contains work packages and (if needed) planning packages.
	\item[STEP 2:] \emph{Schedule the work}. Develop the integrated schedule (often visualized with Gantt charts) that sequences the \ac{wbs} elements, includes key milestones, and forms the backbone of the time-phased budget.
	\item[STEP 3:] \emph{Budget the work}. Assign time-phased budgets to each work package, establish management reserve for in-scope known unknowns, and confirm that the sum of the control accounts equals the contract budget base.
\end{enumerate}
Changes to the \ac{pmb} must be formally controlled and documented. Reasons for changes include:
\begin{itemize}
	\item Contract changes
	\item Internal replanning
	\item Formal reprogramming
\end{itemize}

\subsection{EVM Reviews and Reports}
\srcCite{edo-3-6-1-evm-intro-2025}
\subsubsection{Post-Acceptance Review}
\begin{itemize}
	\item Objective: ensure performance data remain accurate after system acceptance and identify any corrective actions required to reaffirm compliance.
	\item Timing: conducted as needed following system acceptance and prior to the next \ac{ibr}.
	\item Led by the Review Director (typically \ac{dcma}) with membership tailored to the purpose of the review.
	\item Culminates in a formal report prepared by the Review Director.
\end{itemize}
\subsubsection{Initial Compliance Evaluation}
\begin{itemize}
	\item Objective: validate that the contractor’s \ac{evm} system description matches actual practice and satisfies the EVMS criteria.
	\item Timing: executed prior to the initial \ac{ibr} (and subsequently only if needed).
	\item Led by the \ac{dcma} Review Director with cross-functional support as required.
	\item Results documented in a report signed by the Review Director.
\end{itemize}
\subsubsection{Integrated Baseline Review}
\begin{itemize}
	\item Participants: the contractor, \ac{pm} and deputy, \ac{dcma}/\ac{dcaa}/\ac{supship} personnel, and relevant technical staff.
	\item Purpose: joint assessment to verify the realism and accuracy of the \ac{pmb}, confirm the entire technical scope is captured, and ensure resources and schedules are achievable.
	\item Timing: conducted within six months of contract award (or major replanning event) in accordance with the DoD \ac{evm} Implementation Guide and service policy~\autocite{edo-3-6-1-evm-intro-2025}.
\end{itemize}
\subsubsection{Integrated Program Management Data Analysis Report}
This contractually required report delivers cost, schedule, and technical status derived from the contractor’s \ac{evm} system, enabling the \ac{pmo} to identify performance problem areas and emerging risks. It is required on every contract that mandates \ac{evm} and must be delivered at least monthly. The data set is tailored for each contract based on risk, size, and integration considerations, but the canonical formats are:
\begin{enumerate}
	\item \ac{wbs} (most common)
	\item Organizational categories
	\item Program Management Baseline
	\item Staffing
	\item Explanation and Problem Analysis
	\item \ac{ims}
\end{enumerate}
A \ac{csfr} complements the \ac{ipmdar} by providing funding (price) information that reconciles to the cost-focused earned value data~\autocite{edo-3-6-1-evm-intro-2025}.
\subsubsection{Problems}
When using \ac{evm}, the following are indications that a problem exists~\autocite{edo-3-6-2-evm-analysis-2025}:
\begin{itemize}
  \item Use of management reserves
  \item Significant revisions to the \ac{pmb}
  \item Zero variance
  \item Sudden change in trends
  \item Unreasonable \ac{tcpi}
\end{itemize}

\subsection{EVM Data Analysis}
\Fig{\ref{fig:evm_var}} shows the S-curve representation of the \ac{evm} variables; the gaps between the curves are the cost and schedule variances that drive performance assessments~\srcCite{edo-3-6-2-evm-analysis-2025}.
\begin{figure}
	\centering
	\includegraphics[width=0.7\linewidth]{EVM-chart.png}
	\caption[EVM Variances]{EVM chart showing the independent variables and variances. \srcCite{edo-3-6-2-evm-analysis-2025}}
	\label{fig:evm_var}
\end{figure}
\subsubsection{Data Analysis Steps}
\begin{enumerate}
	\item \textbf{Get current status}: capture the latest cost and schedule performance using \ac{cpi}, \ac{spi}, and narrative explanations.
	\item \textbf{Identify trends}: analyze indices and variance trajectories over time to highlight emerging risks and opportunities.
	\item \textbf{Predict completion}: develop independent \ac{eac} and \ac{vac} assessments to forecast final outcomes.
	\item \textbf{Determine management actions}: decide where to apply resources, contract changes, or risk mitigations to recover performance~\srcCite{edo-3-6-2-evm-analysis-2025}.
\end{enumerate}
\subsubsection{Performance Measurement Techniques}
\begin{description}
	\item[Percent Complete:] Applies to long-duration tasks lacking interim milestones; progress is reported as the earned value fraction of total budget.
	\item[Weighted Milestones:] Uses milestone weights for long-duration tasks with clear waypoints, crediting earned value as milestones are completed.
	\item[Percent Start/Percent Finish:] Suitable for short-duration tasks (less than two reporting periods) with credit apportioned at start and finish.
	\item[0/100 Method:] Provides earned value only when the task is complete, offering a conservative measure for discrete, short tasks~\srcCite{edo-3-6-2-evm-analysis-2025}.
\end{description}
\subsubsection{Equations}
Standard cost and schedule performance calculations include~\autocite{edo-3-6-2-evm-analysis-2025}:
\begin{align*}
	\text{CV}          & = \text{EV} - \text{AC}                         \\
	\text{SV}          & = \text{EV} - \text{PV}                         \\
	\text{CPI}         & = \frac{\text{EV}}{\text{AC}}                   \\
	\text{SPI}         & = \frac{\text{EV}}{\text{PV}}                   \\
	\text{CV} \%       & = \left(\frac{\text{CV}}{\text{EV}}\right)\times 100  \\
	\text{SV} \%       & = \left(\frac{\text{SV}}{\text{PV}}\right)\times 100  \\
	\text{\% Complete} & = \left(\frac{\text{EV}}{\text{BAC}}\right)\times 100 \\
	\text{\% Spent}    & = \left(\frac{\text{AC}}{\text{BAC}}\right)\times 100 \\
	\text{EAC}         & = \frac{\text{BAC}}{\text{CPI}}                               \\
  \text{EAC}         & = \text{AC} + \frac{\text{BAC} - \text{EV}}{\left(\frac{\text{EV}}{\text{AC}}\right)_{\text{3 months}}}\\
  \text{EAC}         & = \text{AC} + \frac{\text{BAC} - \text{EV}}{\text{CPI}\times\text{SPI}}\\
  \text{EAC}         & = \text{AC} + \frac{\text{BAC} - \text{EV}}{0.8\,\text{CPI} + 0.2\,\text{SPI}}\\
  \text{VAC}         & = \text{BAC} - \text{EAC}\\
  \text{TCPI}        & = \frac{\text{Work remaining}}{\text{Budget remaining}}\\
  \text{TCPI}_\text{EAC}        & = \frac{\text{BAC} - \text{EV}}{\text{EAC} - \text{AC}}\\
  \text{TCPI}_\text{BAC}        & = \frac{\text{BAC} - \text{EV}}{\text{BAC} - \text{AC}}
\end{align*}

%====================
% End of file
%====================
\ifSubfilesClassLoaded{
	\renewcommand*{\entryname}{\textbf{\color{Modern} Acronym}}
	\renewcommand*{\descriptionname}{\textbf{\color{Modern} Definition}}
	\printnoidxglossary[
		type=\acronymtype,
		title=Acronyms,
		style=long-booktabs
	]}{}
\end{document}
