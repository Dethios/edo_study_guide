% !TEX root = ../main.tex
% !TEX program = lualatex
\documentclass[../main.tex]{subfiles}
\IfSubfilesClassLoaded{\externaldocument{\subfix{../build/main}}}{}
% =============================================================================
% CHAPTER 6: PPBE (PLANNING, PROGRAMMING, BUDGETING, AND EXECUTION)
%
% DESCRIPTION: This chapter provides an overview of the PPBE process,
%              including its timeline, key terms, and the distinction between
%              programming and budgeting. It also covers related topics such
%              as appropriations life-cycle, fiscal law, and reprogramming.
% =============================================================================

\begin{document}
\ifSubfilesClassLoaded{\chapter{EDO Study Guide}}{}
%====================
% Contents here
%====================

% === PPBE ===
\section{PPBE (Planning, Programming, Budgeting, and Execution)}
\minibib
\subsection{What PPBE does}
Aligns strategy to resources across the \ac{fydp}. \emph{Planning} (strategy $\to$ guidance), \emph{Programming} (balanced force/program $\to$ \ac{pom}), \emph{Budgeting} (validated request $\to$ \ac{pb}), \emph{Execution} (obligations/outlays $\to$ performance feedback)~\autocite{edo-3-1-1-ppbe-2025}.

\subsection{Timeline (annual rhythm)}
Figure~\ref{fig:ppbe_timeline} shows the \ac{ppbe} timeline across multiple fiscal years with key milestones during each calendar year.

\begin{figure}[H]
	\centering\color{DarkGray}
	\includegraphics[width=\linewidth]{Images/PPBE Timeline.png}
	\caption[PPBE Timeline for multiple fiscal years]{PPBE Timeline for multiple fiscal years. \srcCite{edo-3-1-1-ppbe-2025}.}
	\label{fig:ppbe_timeline}
\end{figure}

\subsection{Congressional Enactment (Regular Order)}
Figure~\ref{fig:ppbe_congress_timeline} highlights how Congress moves from the \ac{pb} submission through authorization and appropriations when operating on-time without a \ac{cr}. Being able to walk this chart helps bridge \ac{ppbe} milestones with Hill activity.

\begin{figure}[H]
	\centering\color{DarkGray}
	\includegraphics[width=\linewidth]{Images/Congressional Enactment Timeline.png}
	\caption[Congressional Enactment Timeline]{Congressional Enactment Timeline. \srcCite{edo-3-1-2-congressional-enactment-2025}.}
	\label{fig:ppbe_congress_timeline}
\end{figure}

\subsection{Key Terms and Definitions}
Table~\ref{tab:ppbe_key_terms} lists key terms and definitions.

\begin{longtblr}[%
		caption = {Key terms in PPBE},
		label   = {tab:ppbe_key_terms},
		entry   = {Key PPBE Terms},
		remark{Source} = {\tabCite{DoDFMR-Vol3}}%
	]{colspec = {@{} F Q @{}}}
	\toprule
	{Term} & {One-liner}                                              \\
	\midrule
	PY     & Last completed FY: execution look-back and CR baselines. \\
	CY     & Ongoing FY: execution and mid-year reviews.              \\
	BY     & Next FY in submission.                                   \\
	POM    & Service's balanced force/program across the FYDP.        \\
	FYDP   & 5-year program structure and resources.                  \\
	\bottomrule
\end{longtblr}

Other terms and definitions from~\autocite{edo-3-1-1-ppbe-2025}:
\begin{description}
	\item[\textbf{\ac{cape}.}] Provides independent cost assessment and program evaluation.
	\item[\textbf{\ac{pdm}.}] Records \ac{secdef} program decisions at the end of Program Review.
	\item[\textbf{\ac{n8}.}] Builds the Navy \ac{pom}; trades across portfolios with cost/risk realism.
	\item[\textbf{\ac{n9}.}] Validates/advocates warfare requirements; integrates by mission area.
	\item[\textbf{\ac{n91}.}] Manages cross-domain mission integration and architecture; orchestrates \ac{pom} issue papers and mission engineering. \emph{Office designators can shift; verify current subcodes week-of.}
\end{description}

\subsection{Programming v.s.\ Budgeting}
Table \ref{tab:ppbe_prog_vs_budg} compares key differences between Programming and Budgeting.  One way to think about this:  (1) Budgeting is for the money for this year and (2) Programming is for the 5-year \ac{fydp}.

\begin{longtblr}[%
		caption = {Programming v.s.\ Budgeting},
		label   = {tab:ppbe_prog_vs_budg},
		entry   = {Programming v.s.\ Budgeting},
		remark{Source} = {\tabCite{edo-3-1-1-ppbe-2025}}%
	]{colspec = {@{} F L L @{}}}
	\toprule
	{}           & {Programming}                         & {Budgeting}                                  \\
	\midrule
	Purpose      & Build a balanced force across FYDP    & Price/validate an executable BY request      \\
Lead         & N8 with supporting codes/PEOs/SYSCOMs & FMB/Office of the Secretary of Defense (Comptroller)/Office of Management and Budget                    \\
	Key products & POM, PDM                              & Budget estimate, reclama, President's Budget \\
	\bottomrule
\end{longtblr}

\subsection{PPBE Programming}

\paragraph{3-Star v.s.\ 4-Star Reviews:} During \ac{osd}'s Program Review, issues raised on the Services' \acp{pom} move first to a 3-Star Programmers Panel (staff-level, chaired by \ac{dcape}) that vets issue papers, builds options (with offsets), and forwards recommendations. Unresolved or strategy-level trades go to the 4-Star forum, the \ac{dmag} co-led by the Deputy \ac{secdef} (with the \ac{vcjcs}), for senior adjudication. Decisions at this stage are documented as a \ac{pdm} or (in recent cycles) a programmatic \ac{rmd}, which updates the \ac{fydp} and hands off to the Comptroller's Budget Review that culminates in budgetary \acp{rmd}.

\subsection{Appropriations Life-cycle (Colors of Money)}
\label{sec:colors_of_money}
Figure~\ref{fig:ppbe_approps_lifecycle} visualizes when each major appropriation category is Current, Expired, or Cancelled. Pair this with the obligation windows discussion so you can sketch the chart quickly at the board.

\begin{figure}[H]
	\centering\color{DarkGray}
	\includegraphics[width=\linewidth]{Images/Appropriations Life-cycle.png}
	\caption[Colors of Money Timeline]{Colors of Money Timeline. \srcCite{edo-3-1-3-program-funding-2025}.}
	\label{fig:ppbe_approps_lifecycle}
\end{figure}

\subsection{IGT (Intragovernmental Transactions)}
\begin{description}
	\item[\textbf{Definition.}] Orders and collections between federal entities using Treasury's G-Invoicing: 7600A (order), 7600B (agreement/performance), \ac{ipac} for collections~\autocite{Treasury-GInvoicing}.
	\item[\textbf{Why it matters.}] \ac{igt} acceptance \emph{is an obligation} on the customer side and the \emph{revenue recognition driver} for the performing \ac{wcf} activity.
\end{description}

\subsection{Common Ordering Instruments}
\begin{description}
	\item[\textbf{Work Order.}] Internal directive used within a command to control scope/cost/schedule.
	\item[\textbf{Project Order.}] Statutory order for a \emph{definite, specific, and entire} project; obligational on acceptance; performs like a contract between \ac{dod} activities. (41~U.S.C.~\S~6307)~\autocite{USC-6307-ProjectOrder}
	\item[\textbf{\ac{mipr}/IPR (DD Form~448).}] Economy Act order between federal entities; obligational when accepted; performed by the servicing agency. (31~U.S.C.~\S~1535)~\autocite{USC-1535-EconomyAct,Treasury-GInvoicing}
\end{description}

% === FISCAL LAW ANCHORS ===
\subsection{Fiscal Law (Know These Cold)}
\begin{description}
	\item[\textbf{Purpose Statute (``Misappropriation'').}] Funds must be used only for their appropriated purpose (31~U.S.C.~\S~1301(a))~\autocite{USC-1301-Purpose}.
	\item[\textbf{Bona Fide Needs Rule.}] Use current-year appropriations only for legitimate needs of that fiscal year (31~U.S.C.~\S~1502(a))~\autocite{USC-1502-BFN}.
	\item[\textbf{Anti-Deficiency Act.}] Do not obligate/expend in excess of available amounts or before funds are available (31~U.S.C.~\S~1341, \S~1517)~\autocite{USC-1341-ADA}.
\end{description}

% === REPROGRAMMING (WHAT, WHEN, HOW) ===
\subsection{Reprogramming: Moving Resources \emph{After} Enactment}
\paragraph*{Four mechanisms (high level).}
\begin{enumerate}
	\item \textbf{Congressional \ac{pa-reprog}}: Actions above statutory/committee thresholds or affecting congressional special-interest items or new starts. Requires approval from all four defense committees; timing driven by committee cycles~\autocite{DoDFMR-Vol3}.
	\item \textbf{\ac{ir}}: Realignments within an appropriation that do not cross thresholds or trigger congressional interest; used for proper purpose alignment without changing totals. Notice sent to committees~\autocite{DoDFMR-Vol3}.
	\item \textbf{\ac{btr}}: Component-level authority to move funds below set dollar/percent limits within the same appropriation/fiscal year; limits reset annually by appropriations acts/committee guidance~\autocite{DoDFMR-Vol3}.
	\item \textbf{\ac{lt}}: Treasury non-expenditure transfer moving budgetary resources between appropriations/treasury symbols when authorized (e.g., statute or enacted language)~\autocite{DoDFMR-Vol3}.
\end{enumerate}
\hint{\textit{General rules of thumb}---cannot change color-of-money or extend time; must complete before funds' obligation period ends; expired-year accounts limited to valid upward/downward adjustments only~\autocite{DoDFMR-Vol3}.}

% === ALLOCATION FLOW (USD(C) -> ECHELON IV) ===
\subsection{From Enactment to Field Authority}
\begin{enumerate}
	\item Treasury \emph{warrants} the appropriation; \ac{omb} \emph{apportions} on Standard Form~132.
	\item \ac{osd} (Comptroller) issues allocations to components such as the \ac{don}.
	\item \ac{opnav} (\ac{fmb} / \ac{n82}) allocates to \ac{syscom} \acp{bso} (Echelon~II), which suballocate to Warfare Centers (Echelon~III), followed by \emph{allowances/allotments} to the \ac{erp} system~\autocite{DoDFMR-Vol3}.
	\item Commands record commitments, obligations, and expenses, and request outlays via \ac{dfas}.
\end{enumerate}
%====================
% End of file
%====================
\ifSubfilesClassLoaded{
  \RenewDocumentCommand{\entryname}{}{\textbf{\color{Modern} Acronym}}
  \RenewDocumentCommand{\descriptionname}{}{\textbf{\color{Modern} Definition}}
  \printnoidxglossary[
    type=\acronymtype,
    title=Acronyms,
    style=long-booktabs
]}{}
\end{document}
