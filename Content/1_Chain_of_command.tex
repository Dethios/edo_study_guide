% !TEX root = ../main.tex
% !TEX program = lualatex
\documentclass[../main.tex]{subfiles}
\IfSubfilesClassLoaded{\externaldocument{\subfix{../build/main}}}{}
% =============================================================================
% CHAPTER 1: COMMAND CHAINS AND GOVERNANCE
%
% DESCRIPTION: This chapter outlines the operational and acquisition chains
%              of command within the U.S. Navy, highlighting the differences
%              between Title 10 authority and acquisition governance.
% =============================================================================

\begin{document}
\ifSubfilesClassLoaded{\chapter{EDO Study Guide}}{}
%====================
% Contents here
%====================
% === COMMAND CHAINS & GOVERNANCE ===
\section{Command Chains and Governance}
\minibib
\subsection{Operations Chain (Title 10)}

\begin{description}
	\item[\textbf{Chain flow.}] President $\to$\ac{secdef} $\to$\ac{ccdr}. The\ac{cjcs} is the principal military adviser; service chiefs are \emph{not} in the operational chain of Title 10.
	\item[\textbf{Board cue.}] Boards love the contrast with acquisition authority; avoid conflating operational and acquisition chains.
	\item[\textbf{Echelons.}] Counted from the\ac{cno} (e.g.,\ac{cno} is Echelon I,\acp{syscom} are Echelon II, etc.).
\end{description}

Figure \ref{fig:chain_of_command_ops} shows the Title 10 chain of command. \Fig{\ref{fig:chain_of_command_org_chart}} shows the Chain of Command from the President to Echelon III commands.

\begin{figure}[H]
	\centering\color{DarkGray}
	\includegraphics[width=\linewidth]{Command Structure.jpeg}
	\caption[Title 10 Chain of Command]{Title 10 Chain of Command.~\srcCite{edo-2-1-1-command-structures-2025}.}
	\label{fig:chain_of_command_ops}
\end{figure}

\begin{figure}[H]
	\centering\color{DarkGray}
	\includestandalone[width=\linewidth]{TikZ/Org_Chart}
	\caption{Organization Command Structure}
	\label{fig:chain_of_command_org_chart}
\end{figure}

\subsection{Acquisition Governance (Department of the Navy)}

\begin{description}
	\item[\textbf{Policy authority.}]\ac{asnrdanda} is the\ac{sae} for the\ac{don} and sets policy for\acp{peo} and\acp{drpm}~\autocite{SECNAV5400-15D,SECNAV5000-2G}.
	\item[\textbf{Matrix execution.}]\acp{peo} hold program authority;\acp{syscom} (\ac{navsea},\ac{navwar},\ac{navair}) provide the\ac{ta} warrants and workforce;\acp{nwc} deliver research, development, test, and engineering support~\autocite{SECNAV5400-15D}.
\end{description}

Figure \ref{fig:chain_of_command_acq_gov} shows the Navy acquisition governance chain of command.

\begin{figure}[H]
	\centering\color{DarkGray}
	\includegraphics[width=0.7\linewidth]{Images/Navy Acquisition Organization.jpeg}
	\caption[Acquisition Governance Chain of Command]{Acquisition Governance Chain of Command.~\srcCite{edo-2-1-1-command-structures-2025}.}
	\label{fig:chain_of_command_acq_gov}
\end{figure}

\subsection{Approving Authorities}

\begin{figure}[H]
	\centering\color{DarkGray}
	\includegraphics[width=0.7\linewidth]{Images/TA vs PA.png}
	\caption[Technical Authority v.s.\ Program Authority]{Technical Authority v.s.\ Program Authority alignment.~\srcCite{edo-2-1-5-technical-authority-2025}}
	\label{fig:chain_of_command_auth_roles}
\end{figure}

Figure \ref{fig:chain_of_command_auth_roles} contrasts the roles of\ac{ta} and\ac{pa} in Navy acquisition.
%====================
% End of file
%====================
\ifSubfilesClassLoaded{
  \renewcommand*{\entryname}{\textbf{\color{Modern} Acronym}}
  \renewcommand*{\descriptionname}{\textbf{\color{Modern} Definition}}
  \printnoidxglossary[
    type=\acronymtype,
    title=Acronyms,
    style=long-booktabs
]}{}
\end{document}


