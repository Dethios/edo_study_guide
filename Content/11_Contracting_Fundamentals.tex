% !TEX root = ../main.tex
% !TEX program = lualatex
\documentclass[../main.tex]{subfiles}
\ifSubfilesClassLoaded{%
	\externaldocument{\subfix{../build/main}}%
}{}
% =============================================================================
% CHAPTER 11: CONTRACTING FUNDAMENTALS
%
% DESCRIPTION: This chapter provides an introduction to contracting
%              fundamentals, including the essential elements of a binding
%              contract, the FAR system, and the roles of different
%              contracting officers.
% =============================================================================
\begin{document}
\ifSubfilesClassLoaded{\chapter{EDO Study Guide}}{}
%====================
% Contents here
%====================

% === INTRO TO CONTRACTING (EDO 3.2.1) ===
\section{Intro to Contracting Fundamentals}
\minibib
\subsection{Why the Navy Uses Contracts}
\begin{itemize}
	\item Contracts create a legally enforceable relationship between the Government and industry partners, defining rights and responsibilities for each party.
	\item Written terms provide structure for changes via agreed conditions, protecting both the Navy and the Contractor throughout acquisition actions.
	\item Binding agreements ensure accountability for cost, schedule, and performance outcomes that cannot be achieved through informal arrangements.
\end{itemize}

\subsection{Essential Elements of a Binding Contract}
\begin{description}
	\item[\textbf{Mutual Assent.}] Requires an offer and acceptance with a meeting of the minds on all material terms such as scope, price, quantity, and delivery.
	\item[\textbf{Consideration.}] Both sides exchange value (e.g., payment, performance, schedule relief) that courts will evaluate for adequacy in Government contracting.
	\item[\textbf{Capacity.}] Parties must be legally competent to enter the agreement; lack of capacity (e.g., incapacity, lack of authority) undermines enforceability.
	\item[\textbf{Lawful Purpose.}] The contract must pursue a legal objective; agreements for unlawful acts are void.
\end{description}

\subsection{Government Relationship to Contractors}
\begin{description}
	\item[\textbf{Transactional.}]  Each party enters with distinct incentives (contractor profit versus government outcomes); avoid unauthorized commitments.
	\item[\textbf{Professional.}]  Both teams must understand the contract and collaborate to meet requirements.
	\item[\textbf{Collaborative.}] Contractors bring technical expertise while programmatic decision-making remains inherently governmental.
	\item[\textbf{Constrained.}]  Interactions are bound by ethics rules, conflicts-of-interest standards, and contract clauses.
\end{description}

\subsection{FAR System and NAVSEA Overlays}
\begin{description}
\item[\textbf{FAR.}] Government-wide acquisition regulation establishing policy, procedures, and contract clauses~\autocite{FAR}.
\item[\textbf{DFARS.}] \ac{dod}-level supplement tailoring FAR provisions for defense-unique requirements, including competition, source selection, and industrial base policy~\autocite{DFARS}.
\item[\textbf{NMCARS.}] Department of the Navy supplement that adds Navy-specific directives such as approval thresholds, peer reviews, and templates~\autocite{NMCARS}.
\item[\textbf{SOM Chapter 3.}] \ac{navsea} guidance for Supervisors of Shipbuilding on contract formation, modification, and surveillance~\autocite{SOM-Ch3-2023}.
\item[\textbf{NAVSEA Source Selection Guide.}] Standardizes competitive source selection procedures, roles, and documentation for Sea Systems Command procurements~\autocite{NAVSEA-Source-Selection-Guide-2022}.
\item[\textbf{NAVSEA Contracts Handbook.}] Practical desk reference covering policy interpretations, clause usage, and best practices for \ac{navsea} contracting professionals~\autocite{NAVSEA-Contracts-Handbook-2023}.
\end{description}

\subsection{Types of Contracting Officers (Know All Three)}
\begin{description}
	\item[\textbf{\ac{pco}}.] Leads acquisition planning through award; signs contracts and bilateral modifications on behalf of the Government.
	\item[\textbf{\ac{aco}}.] Oversees post-award administration, surveillance, and contractor performance; typically assigned via \ac{dcma} or \ac{navsea} field offices.
	\item[\textbf{\ac{tco}}.] Manages partial or complete contract terminations, settlement proposals, and equitable adjustments.
\end{description}
All contracting officers must hold a warrant that delineates dollar and authority limits; only a warranted \ac{ko} can obligate the United States~\autocite{FAR}.

\subsection{PM and KO Partnership}
\begin{description}
	\item[\textbf{\ac{pm}}.] Accountable to the Milestone Decision Authority for cost, schedule, and performance; integrates technical authority, requirements, and budget execution across the lifecycle.
	\item[\textbf{\ac{ko}}.] Provides acquisition strategy execution expertise, ensures compliance with statute/regulation, and is responsible for contract integrity and enforceability.
	\item[\textbf{Board Cue}.] \acp{pm} lead program outcomes; \acp{ko} safeguard the contracting instrument. Neither can assume the other's authorities, so coordination before solicitations, negotiations, or modifications is mandatory.
\end{description}

\subsection{Competition in Contracting Act Requirements}
\begin{description}
	\item[\textbf{Full and Open Competition}.] Default posture: all responsible sources may compete~\autocite{USC-3201-CICA,FAR}.
	\item[\textbf{Full and Open After Exclusion of Sources}.] Permits set-asides (e.g., small business, 8(a)) or alternate-source strategies when justified~\autocite{FAR}.
	\item[\textbf{Approval for Other than Full and Open}.] Requires documented justification and senior approval per Subpart~6.3~\autocite[Subpart~6.3]{FAR}.
\end{description}

\subsubsection{Seven Exceptions to Full and Open Competition (Memorize)}
\begin{enumerate}
	\item Only one responsible source will satisfy agency requirements (see \autocite[\S~6.302-1]{FAR}).
	\item Unusual and compelling urgency (see \autocite[\S~6.302-2]{FAR}).
	\item Industrial mobilization; engineering, developmental, or research capability (see \autocite[\S~6.302-3]{FAR}).
	\item International agreement (see \autocite[\S~6.302-4]{FAR}).
	\item Authorized or required by statute (see \autocite[\S~6.302-5]{FAR}).
	\item National security (see \autocite[\S~6.302-6]{FAR}).
	\item Public interest (see \autocite[\S~6.302-7]{FAR}).
\end{enumerate}
\hint{Be ready to cite an example scenario for each exception and the approval level required.}

\subsection{Responsiveness, Responsibility, and Key Determinations}
\begin{description}
	\item[\textbf{Responsiveness} (sealed bidding).] Bid must conform to all material terms of the Invitation for Bids; nonconforming bids are rejected without discussion~\autocite{FAR}.
	\item[\textbf{Responsibility}.] Prospective contractor must possess adequate resources, schedule compliance, performance record, integrity, and necessary systems to receive award (\autocite[\S~9.104]{FAR}).
\end{description}

\subsection{Justification and Approval (J\&A) v.s.\ Determination and Findings (D\&F)}
\begin{description}
	\item[\textbf{J\&A.}] Documents the rationale for other-than-full-and-open competition, identifies the chosen statutory exception, and records approval by the appropriate official. Must be posted to \url{https://sam.gov} after award with required redactions~\autocite{FAR}.
	\item[\textbf{D\&F.}] Formal determination that specific conditions are satisfied before taking an action (e.g., use of special contract types, multiyear contracting); states the findings that support the determination~\autocite{FAR}.
\end{description}
\note{At the \ac{nro}, documents and rational are post on the low and high-side \ac{arc}. We are required to post other-full-and-0pen compeition for five days on the \ac{arc} to allow opportuniy for other contracters to bid.}

\subsection{Who Signs D\&Fs (and When)}
\begin{description}
\item[\textbf{General Rule}.] \autocite[\S~1.704]{FAR} requires the contracting officer to sign D\&Fs when the action is within their delegated authority, unless a higher approval level is specified elsewhere in the regulation or delegation memo.
\item[\textbf{\ac{hca}}.] \autocite[\S\S~16.603-3, 16.504(c)(1)(ii)(D)]{FAR} reserve approval for actions such as issuing a letter contract or awarding a single-award task/delivery \ac{idiq} expected to exceed \$100M; \ac{don} \acp{hca} may redelegate no lower than a flag/Senior Executive~\autocite[5201.707]{NMCARS}.
\item[\textbf{\ac{sae}}.] Multiyear contracting, extraordinary contractual relief, or other actions identified in \autocite[\S~17.105-1]{FAR} and \autocite[\S~217.172]{DFARS} require a D\&F signed by the Service \ac{sae} (\ac{asnrdanda} for the Navy, who is also the \ac{don} Senior Procurement Executive) or a specifically delegated official.
\item[\textbf{Document Content}.] Every D\&F must cite the specific statutory/regulatory authority, describe supporting facts, and state the determination in clear language; expiration dates and any required follow-on reviews must also be included per \autocite[\S~1.707]{FAR} and \autocite[5201.707]{NMCARS} guidance.
\end{description}
%====================
% End of file
%====================
\ifSubfilesClassLoaded{
  \renewcommand*{\entryname}{\textbf{\color{Modern} Acronym}}
  \renewcommand*{\descriptionname}{\textbf{\color{Modern} Definition}}
  \printnoidxglossary[
    type=\acronymtype,
    title=Acronyms,
    style=long-booktabs
]}{}
\end{document}

