% !TEX root = ../main.tex
% !TEX program = lualatex
\documentclass[../main.tex]{subfiles}
\IfSubfilesClassLoaded{\externaldocument{\subfix{../build/main}}}{}
%====================
% File: SECTION_FILE_NAME.tex
%====================
\begin{document}
\ifSubfilesClassLoaded{\chapter{EDO Study Guide}}{}
% The section title is not used for regular content. The title is set in the main document.
% === POLICY (EDO 3.3.1) ===
\section{Acquisition Policy}
\minibib

\subsection{Summary}
\begin{description}
	\item[\textbf{Statute drives authority.}] Title~10 charges the Military Departments to equip the force, while each annual \ac{ndaa} refreshes acquisition authorities and reporting duties---know the latest delegation trail before advising a board~\autocite{edo-3-3-1-acq-policy-players-2025}.
	\item[\textbf{Policy = DoD 5000 series.}] DoDD~5000.01 establishes acquisition principles and senior roles; DoDI~5000.02 operationalizes the \ac{aaf} pathways; DoDI~5000.85 prescribes \ac{mca} execution details~\autocite{DoDD5000-01,DoDI5000-02,DoDI5000-85}.
	\item[\textbf{Regulation stack.}] \ac{far}~\textgreater~\ac{dfars}~\textgreater~\ac{nmcars} translates statute and policy into enforceable contracting rules; SECNAVINST~5000.2G tailors the ac{aaf} for ac{don} programs and makes \ac{asnrdanda} the \gls{sae}~\autocite{FAR,DFARS,SECNAV5000-2G}.
\end{description}

The main drivers that a \ac{pm} must balance is cost, schedule, and performance as shown in \Fig{\ref{fig:cost_schedule_performance}}. The trade space for performance versus cost is shown in \Fig{\ref{fig:trade_space}}, which shows the relation of performance and cost with the threshold and objective requirements.

\begin{figure}[H]
	\centering
	\includegraphics{cost_schedule_performance.png}
	\caption[Cost, Schedule, Performance]{The cost, schedule, and performance triangle that must be
		balanced and traded. \srcCite{edo-3-3-1-acq-policy-players-2025}.}
	\label{fig:cost_schedule_performance}
\end{figure}

\begin{figure}[H]
	\centering
	\includegraphics[width=.8\linewidth]{trade_space.png}
	\caption[Trade Space, Performance v.s.\ Cost]{Trade space of performance v.s.\ cost showing the range of viable options with the region of best ``value.'' \srcCite{edo-3-3-1-acq-policy-players-2025}.}
  \label{fig:trade_space}
\end{figure}

\subsection{Practitioner Steps (Board Prep Focus)}
\begin{enumerate}
	\item Confirm statutory authority and delegation: identify the \ac{dae} (USD(A\&S)) or delegated \ac{mdauthority}, the \ac{sae} (\ac{asnrdanda}), and the resource sponsor accountable for the requirement~\autocite{DoDD5000-01,edo-3-3-1-acq-policy-players-2025,SECNAV5000-2G}.
	\item Select the correct \ac{aaf} pathway (or hybrid) and align entry/exit criteria, statutory reports, and decision forums (MS~A/B/C, \ac{mdd}, \acp{bca}, Congressional notices)~\autocite{DoDI5000-02,DoDI5000-85}.
	\item Map contracting rules to the pathway: \ac{far}/\ac{dfars} clauses, competition requirements, and Tailored Acquisition Strategy approvals~\autocite{FAR,DFARS,edo-3-3-1-acq-policy-players-2025}.
	\item Crosswalk Navy overlays: SECNAVINST~5000.2G, SECNAVINST~5400.15D organization responsibilities, and \ac{peo}/\ac{drpm} charters for technical authority touchpoints~\autocite{SECNAV5000-2G,SECNAV5400-15D}.
	\item Prepare decision documentation: update the Acquisition Strategy / \ac{sep} / \ac{temp}, ensure statutory certifications (Clinger-Cohen, \ac{ppbe} affordability caps) are current, and pre-brief the chain (Program Office~\textrightarrow~\ac{peo}~\textrightarrow~\ac{asnrdanda})~\autocite{DoDI5000-85,edo-3-3-1-acq-policy-players-2025}.
\end{enumerate}

\subsection{Policy Stack and Authorities}
\begin{description}
	\item[\textbf{Title 10, U.S.C.}] Establishes Service responsibilities (Subtitle~C for \ac{don} organization, Subtitle~A Part~V for acquisition management) and empowers annual \acp{ndaa} to adjust acquisition thresholds or pilot authorities~\autocite{edo-3-3-1-acq-policy-players-2025}.
	\item[\textbf{DoD Directives/Instructions.}] DoDD~5000.01 sets acquisition principles, governance forums, and senior leader responsibilities; DoDI~5000.02 implements the \ac{aaf} with six scalable pathways; DoDI~5000.85 gives \ac{mca}-specific statutory requirements (\ac{jroc}, cost reporting, baseline control)~\autocite{DoDD5000-01,DoDI5000-02,DoDI5000-85}.
	\item[\textbf{Regulations.}] \ac{far} and \ac{dfars} governs all federal contracting; \ac{dfars} adds \ac{dod}-specific clauses (e.g., earned value, data rights, cybersecurity); \ac{nmcars} adds \ac{don} policy and \ac{asnrdanda} approval levels~\autocite{FAR,DFARS,NMCARS}.
	\item[\textbf{Service overlays.}] SECNAVINST~5000.2G tailors milestone documentation, Naval \ac{syscom} oversight, and Naval Accelerated Acquisition; SECNAVINST~5400.15D assigns \ac{peo} and \ac{syscom} responsibilities for acquisition program execution~\autocite{SECNAV5000-2G,SECNAV5400-15D}.
	\item[\textbf{Advisory guidance.}] The \ac{dag} captures best practices---it is not directive authority, but boards expect you to know how it informs planning reviews and tailoring memoranda~\autocite{DoDAcqGuidebook}.
\end{description}
\Info{Boards expect you to quote the controlling document \emph{and} state who owns the decision. Memorize the policy ladder: Statute~$\rightarrow$~Directive~$\rightarrow$~Instruction~$\rightarrow$~Regulation~$\rightarrow$~Service overlay.}

\subsection{Acquisition Players and Decision Forums}
\begin{description}
	\item[\textbf{DAE.}] USD(A\&S) chairs the \ac{dab}, is \ac{mdauthority} for ACAT~ID/IAM programs unless delegated, and approves key acquisition policies~\autocite{DoDD5000-01,edo-3-3-1-acq-policy-players-2025}.
	\item[\textbf{SAE.}] \ac{asnrdanda} serves as the \ac{don} \ac{sae}; appoints \acp{peo}, assigns \ac{mdauthority} for ACAT~II and below, and ensures Navy acquisition compliance with \ac{dod} policy~\autocite{SECNAV5000-2G,SECNAV5400-15D}.
	\item[\textbf{Chief of Naval Operations/Resource Sponsor.}] Validates requirements and \ac{ppbe} resourcing, provides integrated warfare/community priorities to \ac{asnrdanda} and the \acp{peo}~\autocite{edo-3-3-1-acq-policy-players-2025}.
	\item[\textbf{PEO/DRPM.}] Executes programs within delegated authorities; maintains acquisition baseline control, briefs \ac{asnrdanda} and \ac{dab}-level forums, and ensures \ac{syscom} technical authority integration~\autocite{SECNAV5400-15D,DoDI5000-85}.
	\item[\textbf{PM.}] Accountable for cost, schedule, performance; leads the \ac{ipt}, maintains statutory certifications, and readies milestone documentation~\autocite{DoDI5000-85,edo-3-3-1-acq-policy-players-2025}.
	\item[\textbf{Governance forums.}] Milestone Decision Reviews, the\ac{dab}, Configuration Steering Boards, Overarching \acp{ipt}, and Navy Program Executive Council reviews provide structured oversight and risk adjudication~\autocite{DoDD5000-01,DoDI5000-85,edo-3-3-1-acq-policy-players-2025}.
\end{description}
\note{Edge case: Rapid acquisition authorities (e.g., \ac{uca}) compress governance. Ensure delegation letters document any waived statutory certifications before you recommend skipping a \ac{dab} or \ac{asnrdanda} review~\autocite{DoDI5000-02}.}

\subsection{Adaptive Acquisition Framework Pathways}
The \ac{aaf} can be visualized in \Fig{\ref{fig:aaf}}. The figure shows all six pathways with their purposes and general flow of the acquisition for each.

\begin{figure}[H]
	\centering
	\includegraphics[width=.85\linewidth]{aaf.png}
	\caption[Adaptive Acquisition Pathways]{Adaptive Acquisition Pathways. \srcCite{DAU-AAF}.}
	\label{fig:aaf}
\end{figure}

The \ac{mca} is the main acquisition pathway that \acp{edo} will operate in. As shown in \Fig{\ref{fig:mca}}, it can be broken down into five phases, four major decision points and three milestones.

\begin{figure}[H]
  \centering
  \includegraphics[width=.75\linewidth]{mca_phases.png}
  \caption[Major Capability Acquisition]{Major Capability Acquisition with phases, milestones, and decisions points. \srcCite{edo-3-3-1-acq-policy-players-2025}}
  \label{fig:mca}
\end{figure}

\begin{description}
	\item[\textbf{Major Capability Acquisition.}] Default for weapon systems; milestone-driven with statutory reports (\ac{sar}, \ac{daes}) and Live-Fire/\ac{temp} requirements~\autocite{DoDI5000-85}.
	\item[\textbf{Middle Tier of Acquisition.}] Rapid prototyping (\leq 5~years to field) and rapid fielding pathways; report to USD(R\&E)/USD(A\&S) and Congress semi-annually~\autocite{DoDI5000-02}.
	\item[\textbf{Software Acquisition.}] Iterative DevSecOps delivery, tailored documentation (Software Acquisition Strategy) and continuous Authority to Operate emphasis~\autocite{DoDI5000-02}.
	\item[\textbf{Business Systems.}] Focuses on business process re-engineering, investment review board certification, and Clinger-Cohen Act compliance~\autocite{DoDI5000-02}.
	\item[\textbf{Defense Urgent Capability.}] Immediate warfighter needs; compressed oversight but still documents affordability and sustainment plans~\autocite{DoDI5000-02}.
	\item[\textbf{Services Acquisition.}] Managed via DoDI~5000.74 (referenced in DoDI~5000.02) with governance through functional domain leads; consider when an availability or modernization effort is better treated as a service~\autocite{DoDI5000-02,edo-3-3-1-acq-policy-players-2025}.
\end{description}
\Info{Pathways can be combined (e.g., prototype under Middle-Tier then transition to \ac{mca}). Boards ask why your chosen path best fits cost/schedule urgency.}

\subsection{Navy Overlays and Touchpoints}
\begin{description}
	\item[\textbf{SECNAVINST 5000.2G.}] Implements \ac{asnrdanda} Program Decision Meetings, requires risk-based tailoring plans, and codifies \ac{don}-level documentation (Acquisition Plan, \ac{lcsp}, POA\&M)~\autocite{SECNAV5000-2G}.
	\item[\textbf{SECNAVINST 5400.15D.}] Aligns \acp{peo} and \acp{syscom}, clarifies technical authority (NAVSEA 05, NAVWAR 5.0) integration, and designates Deputy \ac{asnrdanda} portfolio responsibilities~\autocite{SECNAV5400-15D}.
	\item[\textbf{Resource Sponsor coordination.}] \ac{opnav} N9, N4, or equivalent sponsor must endorse requirements and \ac{ppbe} positions before \ac{asnrdanda} decisions~\autocite{edo-3-3-1-acq-policy-players-2025}.
\end{description}
\note{When \ac{secnav} policy conflicts with older \ac{dod} memoranda, defer to the most current directive from the higher authority unless \ac{asnrdanda} grants written tailoring. Document the rationale in the Acquisition Strategy.}

%====================
% End of file
%====================
\ifSubfilesClassLoaded{
	\renewcommand*{\entryname}{\textbf{\color{Modern} Acronym}}
	\renewcommand*{\descriptionname}{\textbf{\color{Modern} Definition}}
	\printnoidxglossary[
		type=\acronymtype,
		title=Acronyms,
		style=long-booktabs
	]}{}
\end{document}


