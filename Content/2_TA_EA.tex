% !TEX root = ../main.tex
% !TEX program = lualatex
\documentclass[../main.tex]{subfiles}
\IfSubfilesClassLoaded{\externaldocument{\subfix{../build/main}}}{}
% =============================================================================
% CHAPTER 2: TECHNICAL AUTHORITY & ENGINEERING AGENTS
%
% DESCRIPTION: This chapter defines Technical Authority (TA) and the roles of
%              various Engineering Agents (EAs) such as ISEA, DA, and AEA.
%              It outlines their responsibilities, the source of their
%              authority, and how they interact in different scenarios.
% =============================================================================

\begin{document}
\ifSubfilesClassLoaded{\chapter{EDO Study Guide}}{}
%====================
% Contents here
%====================

% === TECHNICAL AUTHORITY & ENGINEERING AGENTS ===
\section{Technical Authority and Engineering Agents}
\minibib
\subsection{What Technical Authority Is}

\begin{description}
	\item[\textbf{\ac{ta}.}] Independent engineering authority to set, maintain, and certify conformance to technical standards and baselines across the lifecycle~\autocite{SECNAV5400-15D}.
	\item[\textbf{\ac{uta}.}] Executed at the\ac{syscom} level (e.g.,\ac{navsea}); resides with the Commander and flows through the Chief Engineer via a formal warranting system~\autocite{edo-2-1-5-technical-authority-2025}.
	\item[\textbf{Delegated\ac{ta}.}] Warranted\acp{twh} (\emph{Lead/Local/Warranted}\ac{ta}) receive domain-specific authority across hull, mechanical, and electrical systems, combat systems, cybersecurity, and airworthiness or seaworthiness~\autocite{edo-2-1-5-technical-authority-2025}.
	\item[\textbf{Core\ac{ta} duties.}] Must-know responsibilities:
	      \begin{enumerate}
		      \item Establish and approve technical requirements, standards, and certification criteria;
		      \item Control the authoritative technical baseline (drawings, specs, interface control);
		      \item Adjudicate departures and waivers from specification and safety-critical changes; and
		      \item Certify readiness, readiness for service, and technical acceptability.
	      \end{enumerate}
	\item[\textbf{Independence.}]\ac{ta} remains separate from cost, schedule, and programmatic authority to protect warfighter safety and mission assurance~\autocite{SECNAV5400-15D}.
\end{description}

\subsection{What EAs Do (and Do Not Do)}

\acp{ea} perform lifecycle engineering under\ac{ta} governance. They are \emph{not}\ac{ta} unless they hold a\ac{ta} warrant~\autocite{edo-2-1-5-technical-authority-2025}. Common EA roles include:
\begin{description}
	\item[\textbf{\ac{isea}.}] Lifecycle systems engineering for \emph{fielded} systems (e.g., fleet introduction, distance support, troubleshooting/\ac{casrep} support, maintenance planning, obsolescence, technical manuals/data, configuration of the in-service baseline).
	\item[\textbf{\ac{da}.}] Develops detailed design and configuration documentation for a system or platform (e.g., ship detail design,\acp{icd}, drawings,\acp{tdp}) to\ac{ta}-approved standards.
	\item[\textbf{\ac{aea}.}] Produces alteration packages (e.g., installation drawings, test procedures, logistics updates) and supports fleet introduction for ship or system changes.
	\item[\textbf{\ac{sia}.}] Orchestrates system-of-systems integration (e.g., interfaces, interoperability, cybersecurity in the integration space) across combat system elements and platforms.
	\item[\textbf{\ac{tda}.}] Issues and maintains technical work direction for installations and maintenance (e.g.,\ac{twd}, local instructions) consistent with the\ac{ta}-approved baseline.
\end{description}

\subsection{Where Engineering Agent Authority Comes From}

\begin{description}
	\item[\textbf{Statutory basis.}] \autocite{SECNAV5400-15D} assigns\ac{don} acquisition responsibilities, directing\acp{syscom} to execute\ac{ta} and to designate Warfare Centers as\acp{ea} that support\acp{peo} and\acp{pm}.
	\item[\textbf{Delegations from\ac{ta}.}]\ac{navsea},\ac{navair}, and\ac{navwar} Chief Engineers issue written warrants or designation letters (Lead/Lab/Local\ac{ta}) that flow requirements to specific Warfare Center codes;\ac{ea} charters reference those warrants~\autocite{edo-2-1-5-technical-authority-2025}.
	\item[\textbf{Execution orders.}]\acp{peo} and\acp{pm} provide technical direction letters, project orders, and statements of work that scope the\ac{ea}'s tasks while preserving independence for specification compliance and certification~\autocite{edo-2-1-4-warfare-centers-2024}.
	\item[\textbf{Accountability.}]\acp{ea} report to the\ac{syscom} Chief Engineer for technical rigor and to the sponsoring\ac{pm} for cost and schedule; loss of a warrant or charter terminates their authority to issue technical documentation~\autocite{edo-2-1-5-technical-authority-2025}.
\end{description}

\subsection{Waterfront Triage: What Engages Whom}
\begin{description}
	\item[\textbf{Spec/drawing nonconformance (build or repair).}] Generate a departure or waiver request for the proper\ac{ta} warrant holder; minor deviations may be approved by Local or Lead\ac{ta}, while major or safety-critical issues go to the Chief Engineer/\ac{uta}.
	\item[\textbf{In-service failure (\ac{casrep}/technical assist).}]\ac{isea} opens a support case, provides immediate workarounds and troubleshooting, coordinates root-cause analysis, updates technical data, and recommends permanent fixes (engineering change or alteration).
	\item[\textbf{Change to configuration.}] If a permanent change is needed, the\ac{aea} develops the alteration package;\ac{sia} validates interfaces;\ac{da} updates drawings and\acp{tdp};\ac{ta} approves and certifies.
	\item[\textbf{Principle to remember.}]\acp{ea} execute engineering;\ac{ta} sets the rules and certifies compliance.
\end{description}
%====================
% End of file
%====================
\ifSubfilesClassLoaded{
  \renewcommand*{\entryname}{\textbf{\color{Modern} Acronym}}
  \renewcommand*{\descriptionname}{\textbf{\color{Modern} Definition}}
  \printnoidxglossary[
    type=\acronymtype,
    title=Acronyms,
    style=long-booktabs
]}{}
\end{document}

