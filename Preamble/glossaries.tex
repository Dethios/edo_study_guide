% =============================================================================
% GLOSSARIES AND ACRONYMS (scrbook-friendly)
%
% - Uses glossaries-extra (no indexer: makenoidxglossaries)
% - Abbrev first use: long (SHORT), thereafter: SHORT
% - Optional: reset acronym "first use" per chapter (toggle below)
% =============================================================================

\usepackage[
  stylemods=all,
  acronym,
  nonumberlist,
  nogroupskip,
  nopostdot,
  % shortcuts=ac,
  toc             % add entries to the Table of Contents (KOMA-friendly)
]{glossaries-extra}
\usepackage{xspace}
% --- Abbreviation style: first use "Long (SHORT)", then "SHORT" -------------
\setabbreviationstyle[acronym]{long-short}
% --- Build mode: no external indexing (fast, good for continuous writing) ---
% Kill auto "smart space" in old glossaries/glossaries-extra
\makeatletter
% If a global autospacing flag exists, turn it off.
\@ifundefined{gls@autospacingfalse}{}{%
  \gls@autospacingfalse % old/new releases that have the flag
}%
% In older releases, \glsdoautoinsert performs the lookahead; neutralize it.
\@ifundefined{glsdoautoinsert}{}{%
  \let\glsdoautoinsert\@gobble % swallow its argument(s) and do nothing
}%
% Ensure the final space emitter does nothing.
\@ifundefined{gls@insertspace}{}{%
  \renewcommand*\gls@insertspace{}%
}%
\makeatother
% \let\ac\gls
% \let\acp\glspl
% \let\Ac\Gls
% \let\Acp\Glspl
% --- Load your acronym definitions ------------------------------------------
\loadglsentries{acronyms.def}

% --- OPTIONAL: reset first-use at each chapter (OFF by default) -------------
% Turn on if you want each chapter to re-introduce acronyms on first use.
\newif\ifResetAcronymsPerChapter
\ResetAcronymsPerChapterfalse
% \ResetAcronymsPerChaptertrue   % <- uncomment to enable per-chapter reset
\ifResetAcronymsPerChapter
  \usepackage{etoolbox}%
  \pretocmd{\chapter}{\glsresetall[acronym]}{}{}%
  \pretocmd{\chapter*}{\glsresetall[acronym]}{}{}%
\fi

% --- PDF string safety (bookmarks, etc.) ------------------------------------
% Ensure glossary-related commands render plain text in PDF strings.
% \providecommand\pdfstringdefDisableCommands[1]{}
% \pdfstringdefDisableCommands{%
%   \def\gls#1{\glsentrytext{#1}}%
%   \def\glspl#1{\glsentryplural{#1}}%
%   \def\acrshort#1{\glsentryshort{#1}}%
%   \def\acrlong#1{\glsentrylong{#1}}%
%   \def\acrfull#1{\glsentrylong{#1}\space(\glsentryshort{#1})}%
% }

% =========================
% Hyperlinked, no-space acronyms
% =========================
% Shared helpers for linked acronym output and smart spacing
\NewDocumentCommand{\edoAcrLink}{m m}{\hyperlink{glo:#1}{#2}}%
\newif\ifEdoAcrAutoSpace
\EdoAcrAutoSpacefalse % set true to enable \xspace-style behavior
\NewDocumentCommand{\edoAcrSpace}{}{\ifEdoAcrAutoSpace\xspace\fi}%
\NewDocumentCommand{\edoAcrRender}{m m m m}{%
  \IfBooleanTF{#1}{%
    \edoAcrLink{#2}{#3{#2}}%
  }{%
    \ifglsused{#2}{%
      \edoAcrLink{#2}{#3{#2}}%
    }{%
      \glsadd{#2}%
      \glsunset{#2}%
      \edoAcrLink{#2}{#4{#2}~(#3{#2})}%
    }%
  }%
  \xspace%
}%

% \ac{<lab>}  : first use -> long (SHORT), later -> SHORT; linked; usage tracked
% \ac*{<lab>} : always SHORT; linked; no usage change
\NewDocumentCommand{\ac}{s m}{%
  \edoAcrRender{#1}{#2}{\glsentryshort}{\glsentrylong}%
}%

% Plural
\NewDocumentCommand{\acp}{s m}{%
  \edoAcrRender{#1}{#2}{\glsentryshortpl}{\glsentrylongpl}%
}%

% Capitalized variants
\NewDocumentCommand{\Ac}{s m}{%
  \edoAcrRender{#1}{#2}{\Glsentryshort}{\Glsentrylong}%
}%
\NewDocumentCommand{\Acp}{s m}{%
  \edoAcrRender{#1}{#2}{\Glsentryshortpl}{\Glsentrylongpl}%
}%

% For list labels / moving args (optional; linked)
\NewDocumentCommand{\aclabel}{m}{%
  \hyperlink{glo:#1}{\glsentryshort{#1}}%
}%

% --- Printing notes ----------------------------------------------------------
% In main.tex you already use:
% \printnoidxglossary[type=\acronymtype, title=List of Acronyms, numberedsection, style=long-booktabs]
% You can add 'toctitle=...' there if you want a different ToC title than the heading.
